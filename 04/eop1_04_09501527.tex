% ファイル先頭から\begin{document}までの内容(プレアンブル)については,
% 基本的に { } の中を書き換えるだけでよい.
\documentclass[autodetect-engine,dvi=dvipdfmx,ja=standard,
               a4j,11pt]{bxjsarticle}

%%======== プレアンブル ============================================%%
% 用紙設定:指示があれば,適切な余白に設定しなおす
\RequirePackage{geometry}
\geometry{reset,a4paper}
\geometry{hmargin=25truemm,top=25truemm,bottom=25truemm,footskip=10truemm}
%\geometry{showframe} % 本文の"枠"を確認したければ,コメントアウト

% 設定:図の挿入
% http://www.edu.cs.okayama-u.ac.jp/info/tool_guide/tex.html#graphicx
\usepackage{graphicx}

% 設定:ソースコードの挿入
% http://www.edu.cs.okayama-u.ac.jp/info/tool_guide/tex.html#fancyvrb
\usepackage{fancyvrb}
\renewcommand{\theFancyVerbLine}{\texttt{\footnotesize{\arabic{FancyVerbLine}:}}}

%%======== レポートタイトル等 ======================================%%
% ToDo: 提出要領に従って,適切なタイトル・サブタイトルを設定する
\title{プログラミング演習1 \\
       第4回レポート}

% ToDo: 自分自身の氏名と学生番号に書き換える
\author{氏名: 重近 大智 (SHIGECHIKA, Daichi) \\
        学生番号: 09501527}

% ToDo: レポート課題等の指示に従って適切に書き換える
\date{出題日: 2020年04月20日 \\
      提出日: 2020年05月24日 \\
      締切日: 2020年05月27日 \\}  % 注:最後の\\は不要に見えるが必要.

%%======== 本文 ====================================================%%
\begin{document}
\maketitle
% 目次つきの表紙ページにする場合はコメントを外す
%{\footnotesize \tableofcontents \newpage}

%% 本文は以下に書く.課題に応じて適切な章立てを構成すること.
%% 章=\section,節=\subsection,項=\subsubsection である.

%%%%%%%%%%%%%%%%%%%%%%%%%%%%%%%%%%%%%%%%%%%%%%%%%%%%%%%%%%%%%%%%%%%%%%
\section{概要} \label{sec:abstract}
%%%%%%%%%%%%%%%%%%%%%%%%%%%%%%%%%%%%%%%%%%%%%%%%%%%%%%%%%%%%%%%%%%%%%%

本演習では,名簿管理機能を有するプログラムを,C言語で作成する.
このプログラムは,標準入力から「ID, 氏名, 誕生日, 住所, 備考」からなる
コンマ区切り形式(CSV形式)の名簿データを受け付けて,それらをメモリ中に登録する機能を持つ.
ただし,\%で始まる入力行はコマンド入力と解釈し,登録してあるデータを表示したり整列したりする機能も持つ.

本レポートでは,演習中に取り組んだ課題として,
以下の課題1から課題3についての内容を報告する.
%
\begin{description}
  \item[課題1] 文字列操作の基礎:\verb|subst|関数と\verb|split|関数の実装
  \item[課題2] 構造体や配列を用いた名簿データの定義
  \item[課題3] 標準入力の取得と構文解析
  \item[課題4] \verb|CSV|データ登録処理の実装
  \item[課題5] コマンド中継処理の実装
  \item[課題6] コマンドの実装:\verb|%p|コマンド
  %追記する場合は,本文の記述と矛盾しないように注意すること.
\end{description}
%
また,取り組んだ課題のうち,特に以下の課題については,詳細な考察を行った.
%
\begin{description}
  \item[課題1] 文字列操作の基礎:\verb|subst|関数と\verb|split|関数の実装
  \item[課題3] 標準入力の取得と構文解析
  \item[課題4] \verb|CSV|データ登録処理の実装
  %追記する場合は,本文の記述と矛盾しないように注意すること.
\end{description}


%%%%%%%%%%%%%%%%%%%%%%%%%%%%%%%%%%%%%%%%%%%%%%%%%%%%%%%%%%%%%%%%%%%%%%
\section{プログラムの作成方針}\label{sec:plan}
%%%%%%%%%%%%%%%%%%%%%%%%%%%%%%%%%%%%%%%%%%%%%%%%%%%%%%%%%%%%%%%%%%%%%%

本演習で作成したプログラムが満たすべき要件と仕様として,
「(1) 基本要件」と「(2) 基本仕様」を示す.

\subsubsection*{(1) 基本要件}

\begin{enumerate}
  \setlength{\parskip}{0em} \setlength{\itemsep}{0.25em}  % <-- この一行は項目間の調整
    \item プログラムは,その実行中,少なくとも10,000件の名簿データをメモリ中に保持できるようにすること.
    \item 名簿データは,「ID,氏名,誕生日,住所,備考」を1つのデータとして扱えるようにすること.
    \item プログラムとしての動作や名簿データの管理のために,以下の機能を持つコマンドを実装すること.
    \begin{enumerate} \setlength{\parskip}{0em} \setlength{\itemsep}{0.25em}
        \item プログラムの正常な終了
        \item 登録された名簿データのデータ数表示
        \item 名簿データの全数表示,および,部分表示
        \item 名簿データのファイルへの保存,および,ファイルからの復元
        \item 名簿データの検索と表示
        \item 名簿データの整列
    \end{enumerate}
    \item 標準入力からのユーザ入力を通して,データ登録やデータ管理等の操作を可能とすること.
    \item 標準出力には,コマンドの実行結果のみを出力すること.
\end{enumerate}

\subsubsection*{(2) 基本仕様}

\begin{enumerate}
  \setlength{\parskip}{0em} \setlength{\itemsep}{0.25em}  % <-- この一行は項目間の調整
    \item 名簿データは,コンマ区切りの文字列(\textbf{CSV入力}と呼ぶ)で表されるものとし,
          図\ref{fig:csvdata}に示したようなテキストデータを処理できるようにする.%
    \item コマンドは,\%で始まる文字列(\textbf{コマンド入力}と呼ぶ)とし,表\ref{tab:commands}にあげたコマンドをすべて実装する
    \item 1つの名簿データは,C言語の構造体 (\texttt{struct}) を用いて,構造を持ったデータとしてプログラム中に定義し,使用する
    \item 全名簿データは,“何らかのデータ構造”を用いて,メモリ中に保持できるようにする.
    \item コマンドの実行結果以外の出力は,標準エラー出力に出力する.
\end{enumerate}

%========================= EXAMPLE CSV ================================%
\begin{figure}[b]
\centering % この1行はbegin~endの中を中央寄せにする,というコマンド
% Verbatim environment
% プリアンブルで \usepackage{fancyvrb} が必要.
%   - frame         枠で囲う.single なら,四角で囲われる
%   - xleftmargin   枠の左の余白
%   - xrightmargin  枠の右の余白
%   - gobble        先頭n文字を無視.インデントを無視したい場合に利用する.
%   - fontsize      フォントサイズ指定
%   - numbers       行番号を表示.left なら左に表示.
%   - firstnumber   最初の行番号を指定
\begin{Verbatim}[frame=single, xleftmargin=5mm, xrightmargin=5mm, gobble=4,
                 fontsize=\small, numbers=left, firstnumber=1]
    5100046,The Bridge,1845-11-2,14 Seafield Road Longman Inverness,SEN Unit 2.0 Open
    5100127,Bower Primary School,1908-1-19,Bowermadden Bower Caithness,01955 641225 ...
    5100224,Canisbay Primary School,1928-7-5,Canisbay Wick,01955 611337 Primary 56 3...
    5100321,Castletown Primary School,1913-11-4,Castletown Thurso,01847 821256 01847...
\end{Verbatim}
    \caption{名簿データのCSV入力形式の例.1行におさまらないデータは...で省略した.}
    \label{fig:csvdata}
\end{figure}
%========================= EXAMPLE CSV ================================%

%========================= COMMAND LIST ================================%
\begin{table}[b]  %表の位置は原則として t または b.h や H は使わない.
\centering % この1行はbegin~endの中を中央寄せにする,というコマンド
    \caption{実装するコマンド}
    \label{tab:commands}
    \begin{tabular}{|l|ll|l|}
    \hline
    \textbf{コマンド} & \textbf{意味} &        & \textbf{備考} \\
    \hline\hline
    \verb|%Q|   & 終了                & (Quit)  & \\
    \hline
    \verb|%P n| & 先頭から$n$件表示   & (Print) & $n$が$0$ $\to$ 全件表示,        \\
                &                     &         & $n$が負 $\to$ 後ろから$-n$件表示 \\
    \hline
    *****       &(サンプルのため     & 省略)    & \\
    \hline
    \end{tabular}
\end{table}
%========================= COMMAND LIST ================================%


%%%%%%%%%%%%%%%%%%%%%%%%%%%%%%%%%%%%%%%%%%%%%%%%%%%%%%%%%%%%%%%%%%%%%%
\section{プログラムの説明}\label{sec:capp}
%%%%%%%%%%%%%%%%%%%%%%%%%%%%%%%%%%%%%%%%%%%%%%%%%%%%%%%%%%%%%%%%%%%%%%

プログラムリストは\ref{sec:makep}章に添付している.プログラムは全部で239行からなる.
以下では,\ref{sec:abstract}節の課題ごとに,プログラムの主な構造について説明する.

%--------------------------------------------------------------------%
\subsection{文字列操作の基礎:\texttt{subst}関数と\texttt{split}関数の実装}
%--------------------------------------------------------------------%
% Tips:
% - subsectionなどでは verbが使えない.texttt を使うとよい.
% - 最初の段落で,この節で説明する関数等の名前や場所などを説明する.
% - 以降の段落で,最初の段落で示した,各関数等に関する詳細を説明する.

まず,汎用的な文字列操作関数として,
\verb|subst()|関数を44--58行目で宣言し,
\verb|split()|関数を60--77行目で宣言している.
また,これらの関数で利用するために,\verb|<stdio.h>|というヘッダファイルをインクルードする.

\verb|subst(str, C1, C2)|関数は,\verb|str|が指す文字列中の,文字\verb|C1|を文字\verb|C2|に置き換える.
プログラム中では,\verb|get_line()|関数内の\verb|fgets()|関数で文字列の入力を受けるとき,末尾に付く改行文字を\verb|NULL|文字で置き換えるために使用している.呼び出し元には,文字を置き換えた回数を戻り値として返す.

\verb|split(str, ret[], sep, max)|関数は,他関数から渡された文字列中に文字変数\verb|sep|の文字に一致する文字があった場合,該当文字を\verb|NULL|文字で置き換え,該当文字の次の文字が格納されているメモリのアドレスを\verb|ret[]|に書き込む.なお,\verb|ret[0]|に格納されるアドレスの値は,\verb|split()|関数が呼び出された際の\verb|str|の値である.以降,\verb|sep|の文字に一致する文字があった場合,\verb|ret[]|の添字を1ずつ増やしながらアドレスを格納していく.呼び出し元には,アドレスが格納されている\verb|ret[]|の内,添字が最も大きいものの添字を戻り値として返す.

%--------------------------------------------------------------------%
\subsection{構造体や配列を用いた名簿データの定義}
%--------------------------------------------------------------------%

本名簿管理プログラムでは,構造体の配列を名簿データとして扱う.
9--14行目で,\verb|date|構造体を定義し,
16--23行目で,\verb|profile|構造体を定義している.
この\verb|profile|型の変数1つが,名簿データ1つに相当する.
そして,41行目の\verb|profile_data_store|変数で,全名簿データを管理し,
42行目の\verb|int|型変数\verb|profile_data_nitems|で,名簿データの個数を管理する.
\verb|date|構造体の定義にあたっては,年,月,日に分けて情報を管理できるよう,3つの\verb|int|型変数を用意している.
\verb|profile|構造体は,その要素に\verb|date|構造体を含む.

\verb|profile|構造体の各要素について説明する.
まず,IDを格納するための\verb|int|型変数\verb|id|である.これは\verb|int|型変数の最小値\verb|-2147483648|と最大値\verb|2147483647|の間で整数値を格納できる.
次に,名前を格納するための\verb|char|型の配列\verb|name|である.これは\verb|NULL|文字を含めて最大70文字の文字列を格納できる.
そして,\verb|date|型の変数\verb|birthday|である.これは前述したように年,月,日の3つの\verb|int|型のメンバからなる変数である.
次に,住所を格納するための\verb|char|型の配列\verb|address|である.文字列の配列\verb|name|同様,\verb|NULL|文字を含めて最大70文字の文字列を格納できる.
最後に,備考を格納するための\verb|char|型の配列\verb|biko|である.また,入力情報に備考がない場合,この変数は\verb|NULL|となる.

%--------------------------------------------------------------------%
\subsection{標準入力の取得と構文解析}
%--------------------------------------------------------------------%

標準入力を取得するための\verb|get_line()|関数は79--85行目で宣言している.構文解析のための\verb|parse_line()|関数は,87--114行目で宣言している.標準入出力のため,\verb|<stdio.h>|というヘッダファイルをインクルードする.

\verb|get_line(line)|関数は,標準入力\verb|stdin|を\verb|fgets()|関数で取得し,1024文字以上を越えた場合は,次の行として処理を行うことでバッファオーバーランを防止している.標準入力が\verb|NULL|または,制御文字\verb|ESC|を1文字目に入力した場合は,呼び出し元に戻り値$0$を返す.それ以外の場合,\verb|subst()|関数で末尾の改行文字を\verb|NULL|文字に置き換えた後,呼び出し元に戻り値$1$を返す.

\verb|parse_line(line)|関数は,他関数からの文字列配列をポインタ\verb|line|で取得し,入力内容が特定のコマンドとその引数であるか,単なる文字列であるかを判定し,それぞれ\verb|exec_command()|関数を呼び出すか,\verb|new_profile()|関数を呼び出す.\verb|get_line()|関数で入力された入力文字列の1文字目が\verb|%|である場合,2文字目の文字を文字変数\verb|cmd|に代入,4文字目以降の文字列をポインタ\verb|param|で参照できるようにする.入力文字列がコマンドで引数が与えられなかった場合,ポインタparamは文字列\verb|(Null Parameter)|を参照し,ポインタparamが何も指していない状態を防止している.\verb|exec_command(cmd, *param)|関数が呼び出す一部の自作関数では引数が必要となるので,この処理を行う.必要に応じて\verb|exec_command()|関数内でキャストを行ってから,コマンドの処理をする関数に引数を送る.

%--------------------------------------------------------------------%
\subsection{\texttt{CSV}データ登録処理の実装}
%--------------------------------------------------------------------%

\verb|new_profile()|関数で,\verb|profile|型のグローバル変数\verb|profile_data_store[10000]|に\verb|CSV|データの入力情報を登録する.何番目のデータとして入力情報の登録を行うかは,\verb|int|型のグローバル変数\verb|profile_data_nitems|によって指定する.\verb|profile_data_nitems|は\verb|new_profile()|関数を呼び出す際にインクリメントされるため,重複なくデータの登録が行われる.



%--------------------------------------------------------------------%
\subsection{コマンド中継処理の実装}
%--------------------------------------------------------------------%



%--------------------------------------------------------------------%
\subsection{コマンドの実装:\texttt{\%p}コマンド}
%--------------------------------------------------------------------%



%%%%%%%%%%%%%%%%%%%%%%%%%%%%%%%%%%%%%%%%%%%%%%%%%%%%%%%%%%%%%%%%%%%%%%
\section{プログラムの使用法と実行結果}\label{sec:howresult}
%%%%%%%%%%%%%%%%%%%%%%%%%%%%%%%%%%%%%%%%%%%%%%%%%%%%%%%%%%%%%%%%%%%%%%

本プログラムは名簿データを管理するためのプログラムである.
CSV形式のデータと \% で始まるコマンドを標準入力から受け付け,
処理結果を標準出力に出力する.
入力形式の詳細については,\ref{sec:plan}節で説明した.

% 実行結果を得るために利用した,実行環境のOSを必ず書くこと.
プログラムは,Cent OS 7.6.1810(Core) で動作を確認しているが,
一般的な UNIX で動作することを意図している.
なお,以降の実行例における,行頭の\verb|$|記号は,
Cent OS 7.6.1810(Core) におけるターミナルのプロンプトである.

まず,\verb|gcc|でコンパイルすることで,プログラムの実行ファイルを生成する.
ここで,\verb|-Wall|とは通常は疑わしいものとみなされることのない構文に関して警告を出力するためのオプションであり,
\verb|-o|とは出力ファイル名を指定するオプションである.
これらのオプションをつけることで,疑わしい構文を発見し,任意の出力ファイル名を指定することができる.

{\fontsize{10pt}{11pt} \selectfont
 \begin{verbatim}
   $ gcc program1.c -o program1 -Wall
 \end{verbatim}
}
%% 注:行送りの変更は"指定箇所を含む段落”に効果があらわれる.
%%     fontsizeコマンドを用いて,行送りを変える場合は,
%%     その {...} の前後に空白行を入れ,段落を変えるようにすること.
%%     なお,行先頭がコメントから始まる行は空白行とは扱われない.

次に,プログラムを実行する.
以下の実行例は,プログラム実行中の動作例を模擬するため,
任意の\verb|csv|ファイルを標準入力のリダイレクションにより与えることで,実行する例を示している\cite{www:label1}.
通常の利用においては,キーボードから文字列を入力してもよい.

{\fontsize{10pt}{11pt} \selectfont
 \begin{verbatim}
   $ ./program1 < csvdata.csv
 \end{verbatim}
}

プログラムの出力結果として,CSVデータの各項目が読みやすい形式で出力される.
例えば,下記の \verb|test.csv| に対して,

{\fontsize{10pt}{11pt} \selectfont
 \begin{verbatim}
  Microsoft,Windows,7
  %Q
  y
 \end{verbatim}
}

\noindent % noindentとはここでは段落を変えない(一字下げをしない)というコマンド.
以下のような出力が得られる.

{\fontsize{10pt}{11pt} \selectfont
 \begin{verbatim}
  line number:1
  input:"Microsoft,Windows,7"
  split[0]:"Microsoft"
  split[1]:"Windows"
  split[2]:"7"

  Do you want to quit?(y/n)
  quit success.
 \end{verbatim}
}

まず,入力データについて説明する.
入力中の最初の1行で,1つのCSVデータを登録している.
CSVデータは自動的に分割されて表示される.
\verb|%Q|コマンドによる終了を実施する際に確認メッセージがあるため,最後にyをcsvファイルから入力している.

%CSVデータは,・・・で示したように,7つの項目からなる.
% 注:上1行の「表1」の部分は,\refを用いて適切に参照すること.
%3行目の\%S3 は,これまでの入力データを3番目の項目(生年月日)で
%ソートすることを示している.
%4行目の\%P0 は,入力した項目の全ての項目 ($1$--$7$) を
%表示することを示している.

%(※サンプルのため省略)



%%%%%%%%%%%%%%%%%%%%%%%%%%%%%%%%%%%%%%%%%%%%%%%%%%%%%%%%%%%%%%%%%%%%%%
\section{考察}
%%%%%%%%%%%%%%%%%%%%%%%%%%%%%%%%%%%%%%%%%%%%%%%%%%%%%%%%%%%%%%%%%%%%%%

\ref{sec:capp}章のプログラムの説明,および,\ref{sec:howresult}章の使用法と実行結果から,
演習課題として作成したプログラムが,
\ref{sec:abstract}章で述べた基本要件と基本仕様のいずれも満たしていることを示した.
ここでは,個別の課題のうち,以下の3つの項目について,考察を述べる.
% 注:\ref コマンドを用いて適切に参照すること.

\begin{enumerate}
\setlength{\parskip}{2pt} \setlength{\itemsep}{2pt}
    \item 文字列操作の基礎:\texttt{subst}関数と\texttt{split}関数の実装
    \item 標準入力の取得と構文解析
    \item \texttt{CSV}データ登録処理の実装
\end{enumerate}

%--------------------------------------------------------------------%
\subsection{「文字列操作の基礎:\texttt{subst}関数と\texttt{split}関数の実装」に関する考察}
%--------------------------------------------------------------------%

\verb|subst()|関数が,他関数から文字列の配列を受け取ることを想定して,ポインタを使用して他関数内の配列を参照できるようにした.また,入力文字列の途中に\verb|NULL|文字が出現することは標準入力では起こらないため,文字置き換えのループ処理の継続条件に\verb|NULL|文字を用いることで,確実に入力文字列の終端である\verb|NULL|文字手前まで文字の置き換えを行えるようにした.\verb|c1|と\verb|c2|に同じ文字が入力された場合,文字の置き換え処理は実質行われないに等しい.50行目に\verb|break|文を書くことにより,文字の置き換えと置き換えられた文字のカウントを行わず,処理の簡略化及び高速化ができた.


%--------------------------------------------------------------------%
\subsection{「標準入力の取得と構文解析」に関する考察}
%--------------------------------------------------------------------%

\verb|get_line()|関数内の\verb|fgets()|関数で標準入力を取得する際,直接入力では\verb|Ctrl+D|で標準入力を\verb|NULL|にすることができるが,デバッグ用の機能として\verb|ESC|を1文字目に入力することにより,入力待ちを終了できるように82行目に戻り値の条件を追加した.\verb|ESC|は制御文字であり,ファイルリダイレクションによりファイルから入力されることはないため,これを追加した\cite{www:label2}.ファイルリダイレクションを\verb|fgets()|関数により取得した文字列は,終端が改行文字になってしまうため,\verb|subst()|関数を呼び出し,改行文字を\verb|NULL|文字に置き換える処理も含めている.この処理を行うことで,\verb|subst()|関数のループ処理の継続条件,\verb|split()|関数のループ処理の継続条件や\verb|printf()|関数による文字列の出力などで文字列が扱いやすくなった.\verb|parse_line()|関数では,例外が発生する可能性が多いので,3文字目以降に入力がない場合や3文字目に誤って入力してしまった場合などの対策が必要だと考えた.3文字目以降に入力がない場合を考慮し,仮の\verb|(Null Parameter)|を設定している.また\verb|parse_line()|関数自体が,他関数から文字列の配列の先頭アドレスを受け取り,ポインタ\verb|line|に格納するが,他関数にポインタ\verb|line|をそのまま渡す可能性もあったため,ポインタの値自体をインクリメントしたり,デクリメントしたりすることは避けるとともに,表記を\verb|subst()|関数や\verb|split()|関数と統一している.

%--------------------------------------------------------------------%
\subsection{「\texttt{CSV}データ登録処理の実装」に関する考察}
%--------------------------------------------------------------------%

(※サンプルのため省略)



%%%%%%%%%%%%%%%%%%%%%%%%%%%%%%%%%%%%%%%%%%%%%%%%%%%%%%%%%%%%%%%%%%%%%%
\section{感想}\label{sec:review}
%%%%%%%%%%%%%%%%%%%%%%%%%%%%%%%%%%%%%%%%%%%%%%%%%%%%%%%%%%%%%%%%%%%%%%

毎回の講義でプログラムのソースコードを追記していくとき,どの関数を上から順に書けばいいかを考えていたが,煩雑化してしまうため,関数プロトタイプ宣言をすることで見やすくなり作業がしやすくなった\cite{book:meikai}.今までの自作プログラムでは,関数プロトタイプ宣言が必要になるほど自作関数を用意することがなかったので,実際のプログラミングの経験になった.また,プログラム全体を通して表記のゆれが少なくなるように,多くの自作関数において,ポインタの指し先をずらすために\verb|int|型変数\verb|i|を使用したり,各関数内で似たような役割を持つ変数の変数名を統一したりした.今回の課題プログラムの作成を通して,ポインタへの理解が一層深まり,ポインタのポインタやポインタの配列に関して理解することができたと思う\cite{www:label3,www:label4}.\verb|printf()|関数の書式指定\verb|%s|の引数として,任意のアドレスを代入するが,ポインタの配列\verb|*ret[]|の場合,どのような形で書けばいいのか悩んだ.\verb|printf()|関数の書式指定\verb|%c|の引数では,引数が文字コードなので,ポインタを利用して引数を指定する場合\verb|*|を付ける必要があった.一方書式指定\verb|%s|の場合は,引数がアドレスのため\verb|*|は必要なく,単に\verb|ret[]|と書くので混乱した.

%%%%%%%%%%%%%%%%%%%%%%%%%%%%%%%%%%%%%%%%%%%%%%%%%%%%%%%%%%%%%%%%%%%%%%
\section{作成したプログラム}\label{sec:makep}
%%%%%%%%%%%%%%%%%%%%%%%%%%%%%%%%%%%%%%%%%%%%%%%%%%%%%%%%%%%%%%%%%%%%%%

作成したプログラムを以下に添付する.
なお,\ref{sec:abstract}章に示した課題については,
\ref{sec:howresult}章で示したようにすべて正常に動作したことを付記しておく.

% Verbatim environment
% プリアンブルで \usepackage{fancyvrb} が必要.
%   - numbers           行番号を表示.left なら左に表示.
%   - xleftmargin       枠の左の余白.行番号表示用に余白を与えたい.
%   - numbersep         行番号と枠の間隙 (gap).デフォルトは 12 pt.
%   - fontsize          フォントサイズ指定
%   - baselinestretch   行間の大きさを比率で指定.デフォルトは 1.0.
作成したプログラムのソースコード(239行)\label{func}
\begin{Verbatim}[fontsize=\small, baselinestretch=0.8]
     1	#include <stdio.h>
     2	#include <string.h>                                        /*strncpy関数等用*/
     3	#include <stdlib.h>                                        /*exit関数用*/
     4	
     5	#define ESC 27                                             /*文字列ESCをESC
のASCIIコードで置換*/
     6	#define MAX_LINE 1025                                      /*文字配列LINEの
最大入力数の指定用*/
     7	
     8	/*構造体宣言*/
     9	struct date
    10	{
    11	  int y; /*年*/
    12	  int m; /*月*/
    13	  int d; /*日*/
    14	};
    15	
    16	struct profile
    17	{
    18	  int id;               /*ID*/
    19	  char name[70];        /*名前*/
    20	  struct date birthday; /*誕生日(date構造体)*/
    21	  char address[70];     /*住所*/
    22	  char biko[70];        /*備考*/
    23	};
    24	
    25	/*関数プロトタイプ宣言(煩雑化防止)*/
    26	int subst(char *str, char c1, char c2);
    27	int split(char *str, char *ret[], char sep, int max);
    28	int get_line(char *line);
    29	void parse_line(char *line);
    30	void exec_command(char cmd, char *param);
    31	void cmd_quit();
    32	void cmd_check();
    33	void cmd_print();
    34	void cmd_read();
    35	void cmd_write();
    36	void cmd_find();
    37	void cmd_sort();
    38	void new_profile(struct profile *profile_p, char *line);
    39	
    40	/*グローバル変数宣言*/
    41	struct profile profile_data_store[10000];                  /*profile情報を
格納*/
    42	int profile_data_nitems = 0;                               /*profile情報の
保存数を格納*/
    43	
    44	int subst(char *str, char c1, char c2)
    45	{
    46	  int i;                                                   /*forループ用*/
    47	  int c = 0;                                               /*置き換えた文字
数のカウント用*/
    48	  for(i = 0; *(str + i) != '\0'; i++)                      /*入力文字列の
終端に辿り着くまでループ*/
    49	    {
    50	      if(c1 == c2) break;                                  /*見た目上文字列
に変化がないとき*/
    51	      if(*(str + i) == c1)                                 /*(str + i)の
文字がc1の文字と同じとき*/
    52		{
    53		  *(str + i) = c2;                                 /*(str + i)の
文字をc2の文字に置き換える*/
    54		  c++;                                             /*置き換えた文字
を数える*/
    55		}
    56	    }
    57	  return c;                                                /*置き換えた文字
数を戻り値とする.*/
    58	}
    59	
    60	int split(char *str, char *ret[], char sep, int max)
    61	{
    62	  int i;                                                   /*forループ用*/
    63	  int c = 0;                                               /*ポインタの配列
の指定用*/
    64	
    65	  ret[0] = str;                                            /*ret[0]にstr
の先頭アドレスを代入*/
    66	
    67	  for(i = 0; *(str + i) != '\0'&& c < max; i++)            /*cがmaxより小
さいかつ入力文字列の終端に辿り着いていないときループ*/
    68	    {
    69	      if(*(str + i) == sep)                                /*(str + i)が
sepのとき*/
    70		{
    71		  *(str + i) = '\0';                               /*(str + i)に
NULLを代入*/
    72		  c++;
    73		  ret[c] = str + (i + 1);                          /*ret[c]にNULL
文字の"次の"アドレスを代入*/
    74		}
    75	    }
    76	  return c;                                                /*文字列をいく
つに分割したかを戻り値とする*/
    77	}
    78	
    79	int get_line(char *line)
    80	{
    81	  if(fgets(line, MAX_LINE, stdin) == NULL) return 0;       /*入力文字列が
空のとき,0を戻り値とする.入力文字列は1024文字*/
    82	  if(*line == ESC) return 0;                               /*直接入力の
とき,入力文字列を空にできないため,ESCキーの単独入力により0を戻り値とする(デバッグ用)*/
    83	  subst(line, '\n', '\0');                                 /*subst関数に
より,入力の改行文字を終端文字に置き換える*/
    84	  return 1;                                                /*入力文字列が
存在したとき,1を戻り値とする*/
    85	}
    86	
    87	void parse_line(char *line)
    88	{
    89	  char cmd;                                                /*%の次の1文字
を格納用*/
    90	  char *param;                                             /*コマンドの
パラメータとなる文字列へのポインタ用*/
    91	  char *buffer = "(Null Parameter)";                       /*例外処理用*/
    92	
    93	  if(*line == '%')                                         /*入力文字列
の1文字目が%のとき*/
    94	    {
    95	      cmd = *(line + 1);                                   /*cmdに入力
文字列の2文字目の値を代入*/
    96	      if(*(line + 3) != '\0')                              /*パラメータ部
があるとき*/
    97		{
    98		  if(*(line + 2) != ' ')                           /*3文字目が空白
でないとき*/
    99		    {
   100		      param = line + 2;
   101		      printf("引数を要するコマンド入力の場合,3文字目は空白である必要が
あります.\n処理を中止しました.\n\n");
   102		      return;
   103		    }
   104		  else
   105		  param = line + 3;                                /*ポインタline
に3を足したアドレスをポインタparamに代入*/
   106		}
   107	      else param = buffer;                                 /*入力文字列に
パラメータ部が無いとき,文字列"(Null Parameter)"のアドレスをポインタparamに代入*/
   108	      exec_command(cmd, param);
   109	    }
   110	  else                                                     /*入力がコマンド
ではないとき*/
   111	    {
   112	      new_profile(&profile_data_store[profile_data_nitems++] ,line);
   113	    }
   114	}
   115	
   116	void exec_command(char cmd, char *param)
   117	{
   118	  switch (cmd) {
   119	  case 'Q': cmd_quit();   break;
   120	  case 'T': printf("Parameter test:\"%s\"\n", param); break;   /*ポインタparamの参照先から後ろに向かって,NULLまで文字列を表示する(デバッグ用)*/
   121	  case 'C': cmd_check();  break;
   122	  case 'P': cmd_print();  break;
   123	  case 'R': cmd_read();   break;
   124	  case 'W': cmd_write();  break;
   125	  case 'F': cmd_find();   break;
   126	  case 'S': cmd_sort();   break;
   127	  default: fprintf(stderr, "%%%c command is invoked with arg: \"%s\"\n", cmd, param); break;/*エラーメッセージを表示*/
   128	  }
   129	}
   130	
   131	void cmd_quit()
   132	{
   133	  char c;
   134	
   135	  while(1)
   136	    {
   137	      printf("Do you want to quit?(y/n)\n");               /*確認メッセージ*/
   138	      c = getchar();
   139	      getchar();                                           /*getcharでの入力
時に改行文字が残ってしまうため*/
   140	      if(c == 'y')
   141		{
   142		  printf("quit success.\n\n");
   143		  exit(0);
   144		}
   145	      else if(c == 'n')
   146		{
   147		  printf("quit cancelled.\n\n");
   148		  break;
   149		}
   150	    }
   151	}
   152	
   153	void cmd_check()
   154	{
   155	  fprintf(stderr, "Check command is invoked.\n");
   156	}
   157	
   158	void cmd_print()
   159	{
   160	  fprintf(stderr, "Print command is invoked.\n");
   161	}
   162	
   163	void cmd_read()
   164	{
   165	  fprintf(stderr, "Read command is invoked.\n");
   166	}
   167	
   168	void cmd_write()
   169	{
   170	  fprintf(stderr, "Write command is invoked.\n");
   171	}
   172	
   173	void cmd_find()
   174	{
   175	  fprintf(stderr, "Find command is invoked.\n");
   176	}
   177	
   178	void cmd_sort()
   179	{
   180	  fprintf(stderr, "Sort command is invoked.\n");
   181	}
   182	
   183	void new_profile(struct profile *profile_p, char *line)
   184	{
   185	  char *ret[10];
   186	  char *ret2[4];                                           /*誕生日の情報
を分割し,その先頭アドレスを保存*/
   187	  char sep = ',';                                          /*csvファイル
からの入力を想定しているため,カンマ*/
   188	  char sep2 = '-';                                         /*誕生日の入力
文字列にあるハイフンで区切るため,ハイフン*/
   189	  int max = 10;
   190	  int max2 = 4;
   191	  int c, birth_c;
   192	  static int a = 1;                                        /*値をmain関数
終了時まで保持する必要があるため,static int型*/
   193	  
   194	  printf("line number:%d\n", a);                           /*行番号表示(デ
バッグ用)*/
   195	  //printf("input:\"%s\"\n", line);                        /*入力文字列をそ
のまま表示(デバッグ用)*/
   196	  c = split(line, ret, sep, max);                          /*ID,名前などの
情報を分割する*/
   197	  if(c <= 2)                                               /*備考以外の入力
がない場合*/
   198	    {
   199	      printf("情報はID,名前,誕生日,住所,備考の順で入力される必要があり,備考以外
は必須項目です.\n処理を中止しました.\n\n");
   200	      profile_data_nitems--;                               /*処理中止により,
構造体に情報を書き込まないため*/
   201	      return;
   202	    }
   203	  birth_c = split(ret[2], ret2, sep2, max2);               /*誕生日の年,月,
日を分割する*/
   204	  if(birth_c != 2)                                         /*誕生日の年,月,
日を正常に分割できない場合*/
   205	    {
   206	      printf("誕生日は\"年-月-日\"の形で入力される必要があります.\n処理を中止しまし
た.\n\n"); /*年,月,日に分割できない場合,処理を停止*/
   207	      profile_data_nitems--;                               /*処理中止により,
構造体に情報を書き込まないため*/
   208	      return;
   209	    }
   210	
   211	  /*構造体への情報の書き込み処理*/
   212	  profile_p->id = atoi(ret[0]);                            /*IDの書き込み*/
   213	  strncpy(profile_p->name, ret[1], 70);                    /*名前の書き込み*/
   214	  (profile_p->birthday).y = atoi(ret2[0]);                 /*誕生年の書き込み*/
   215	  (profile_p->birthday).m = atoi(ret2[1]);                 /*誕生月の書き込み*/
   216	  (profile_p->birthday).d = atoi(ret2[2]);                 /*誕生日の書き込み*/
   217	  strncpy(profile_p->address, ret[3], 70);                 /*住所の書き込み*/
   218	  if(ret[4] != NULL)
   219	    strncpy(profile_p->biko, ret[4], 70);                  /*備考があるときのみ,
備考の書き込み*/
   220	
   221	  printf("id:\"%d\"\n", profile_p->id);
   222	  printf("name:\"%s\"\n", profile_p->name);
   223	  printf("birthday:\"%d-%d-%d\"\n", (profile_p->birthday).y, 
(profile_p->birthday).m, (profile_p->birthday).d);
   224	  printf("address:\"%s\"\n",profile_p->address);
   225	  printf("biko:\"%s\"\n\n",profile_p->biko);
   226	
   227	  a++;
   228	}
   229	
   230	int main(void)
   231	{
   232	  char LINE[MAX_LINE] = {0};                               /*入力文字列(1行分)は
main関数で管理*/
   233	
   234	  while(get_line(LINE))                                    /*文字配列LINEに文字列
を入力する(get_line関数)*/
   235	    {
   236	      parse_line(LINE);                                    /*入力文字列がある場合,
構文解析を行う(parse_line関数)*/
   237	    }
   238	  return 0;
   239	}

\end{Verbatim}


%%%%%%%%%%%%%%%%%%%%%%%%%%%%%%%%%%%%%%%%%%%%%%%%%%%%%%%%%%%%%%%%%%%%%%
% 参考文献
%   実際に,参考にした書籍等の奥付などを見て書くこと.
%   本文で引用する際は,\cite{book:algodata}のように書けばよい.
%%%%%%%%%%%%%%%%%%%%%%%%%%%%%%%%%%%%%%%%%%%%%%%%%%%%%%%%%%%%%%%%%%%%%%
\begin{thebibliography}{99}
  \bibitem{book:algodata} 平田富雄,アルゴリズムとデータ構造,森北出版,1990.
  \bibitem{book:meikai} 林晴比古,明快入門C,SBクリエイティブ,2013
  \bibitem{www:label1} 入出力のリダイレクションとパイプ,http://web.sfc.keio.ac.jp/~manabu/command/contents/pipe.html,2020.05.14.
  \bibitem{www:label2} IT用語辞典E-Words ASCII文字コード,http://e-words.jp/p/r-ascii.html,2020.05.14
  \bibitem{www:label3} 10-3.ポインタと文字列,http://www9.plala.or.jp/sgwr-t/c/sec10-3.html,2020.05.14.
  \bibitem{www:label4} 4.3 ポインタ配列,http://cai3.cs.shinshu-u.ac.jp/sugsi/Lecture/c2/e\_04-03.html,2020.05.14.%アンダースコアはLaTeXの特殊文字にあたるため,\を挿入している.
  \bibitem{www:label5} strncpy,http://www9.plala.or.jp/sgwr-t/lib/strncpy.html,2020.5.21.
  \bibitem{www:label6} 文字列を数値に変換する - C の部屋,http://www.t-net.ne.jp/~cyfis/c/stdlib/atoi.html,2020.5.21.
  \bibitem{www:label7} IT用語辞典E-Words コアダンプ,http://e-words.jp/w/\%E3\%82\%B3\%E3\%82\%A2\%E3\%83\%80\%E3
\%83\%B3\%E3\%83\%97.html,2020.5.22.%'%'はLaTeXの特殊文字にあたるため,\を挿入している.
  %\bibitem{a} 著者名,書名,出版社,発行年.
  %\bibitem{b} WWWページタイトル,URL,アクセス日.
\end{thebibliography}

%--------------------------------------------------------------------%
%% 本文はここより上に書く(\begin{document}~\end{document}が本文である)
\end{document}
