% ファイル先頭から\begin{document}までの内容(プレアンブル)については,
% 基本的に { } の中を書き換えるだけでよい.
\documentclass[autodetect-engine,dvi=dvipdfmx,ja=standard,
               a4j,11pt]{bxjsarticle}

%%======== プレアンブル ============================================%%
% 用紙設定:指示があれば,適切な余白に設定しなおす
\RequirePackage{geometry}
\geometry{reset,a4paper}
\geometry{hmargin=25truemm,top=25truemm,bottom=25truemm,footskip=10truemm}
%\geometry{showframe} % 本文の"枠"を確認したければ,コメントアウト

% 設定:図の挿入
% http://www.edu.cs.okayama-u.ac.jp/info/tool_guide/tex.html#graphicx
\usepackage{graphicx}

% 設定:ソースコードの挿入
% http://www.edu.cs.okayama-u.ac.jp/info/tool_guide/tex.html#fancyvrb
\usepackage{fancyvrb}
\renewcommand{\theFancyVerbLine}{\texttt{\footnotesize{\arabic{FancyVerbLine}:}}}

%%======== レポートタイトル等 ======================================%%
% ToDo: 提出要領に従って,適切なタイトル・サブタイトルを設定する
\title{プログラミング演習2 \\
       期末レポート}

% ToDo: 自分自身の氏名と学生番号に書き換える
\author{氏名: 重近 大智 (SHIGECHIKA, Daichi) \\
        学生番号: 09501527}

% ToDo: レポート課題等の指示に従って適切に書き換える
\date{出題日: 2020年04月22日 \\
      提出日: 2020年07月28日 \\
      締切日: 2020年07月29日 \\}  % 注:最後の\\は不要に見えるが必要.

%%======== 本文 ====================================================%%
\begin{document}
\maketitle
% 目次つきの表紙ページにする場合はコメントを外す
%{\footnotesize \tableofcontents \newpage}

%% 本文は以下に書く.課題に応じて適切な章立てを構成すること.
%% 章=\section,節=\subsection,項=\subsubsection である.

%%%%%%%%%%%%%%%%%%%%%%%%%%%%%%%%%%%%%%%%%%%%%%%%%%%%%%%%%%%%%%%%%%%%%%
\section{概要} \label{sec:abstract}
%%%%%%%%%%%%%%%%%%%%%%%%%%%%%%%%%%%%%%%%%%%%%%%%%%%%%%%%%%%%%%%%%%%%%%

本演習では,名簿管理機能を有するプログラムを,C言語で作成する.
このプログラムは,標準入力から「ID, 氏名, 誕生日, 住所, 備考」からなる
コンマ区切り形式(CSV形式)の名簿データを受け付けて,それらをメモリ中に登録する機能を持つ.
ただし,\%で始まる入力行はコマンド入力と解釈し,登録してあるデータを表示したり整列したりする機能も持つ.

本レポートでは,演習中に取り組んだ課題として,
以下の課題1から課題10についての内容を報告する.
%
\begin{description}
  \item[課題1] 文字列操作の基礎:\verb|subst|関数と\verb|split|関数の実装
  \item[課題2] 構造体や配列を用いた名簿データの定義
  \item[課題3] 標準入力の取得と構文解析
  \item[課題4] \verb|CSV|データ登録処理の実装
  \item[課題5] コマンド中継処理の実装
  \item[課題6] コマンドの実装:\verb|%p|コマンド
  \item[課題7] \verb|%R|コマンドと\verb|%W|コマンド
  \item[課題8] \verb|%F|コマンド
  \item[課題9] \verb|%S|コマンド
  \item[課題10] 独自コマンドの実装
  %追記する場合は,本文の記述と矛盾しないように注意すること.
\end{description}
%
また,取り組んだ課題のうち,特に以下の課題については,詳細な考察を行った.
%
\begin{description}
  \item[課題1] 文字列操作の基礎:\verb|subst|関数と\verb|split|関数の実装
  \item[課題3] 標準入力の取得と構文解析
  \item[課題4] \verb|CSV|データ登録処理の実装
  \item[課題7] \verb|%R|コマンドと\verb|%W|コマンド
  \item[課題8] \verb|%F|コマンド
  \item[課題9] \verb|%S|コマンド
  %追記する場合は,本文の記述と矛盾しないように注意すること.
\end{description}


%%%%%%%%%%%%%%%%%%%%%%%%%%%%%%%%%%%%%%%%%%%%%%%%%%%%%%%%%%%%%%%%%%%%%%
\section{プログラムの作成方針}\label{sec:plan}
%%%%%%%%%%%%%%%%%%%%%%%%%%%%%%%%%%%%%%%%%%%%%%%%%%%%%%%%%%%%%%%%%%%%%%

本演習で作成したプログラムが満たすべき要件と仕様として,
「(1) 基本要件」と「(2) 基本仕様」を示す.

\subsubsection*{(1) 基本要件}

\begin{enumerate}
  \setlength{\parskip}{0em} \setlength{\itemsep}{0.25em}  % <-- この一行は項目間の調整
    \item プログラムは,その実行中,少なくとも10,000件の名簿データをメモリ中に保持できるようにすること.
    \item 名簿データは,「ID,氏名,誕生日,住所,備考」を1つのデータとして扱えるようにすること.
    \item プログラムとしての動作や名簿データの管理のために,以下の機能を持つコマンドを実装すること.
    \begin{enumerate} \setlength{\parskip}{0em} \setlength{\itemsep}{0.25em}
        \item プログラムの正常な終了
        \item 登録された名簿データのデータ数表示
        \item 名簿データの全数表示,および,部分表示
        \item 名簿データのファイルへの保存,および,ファイルからの復元
        \item 名簿データの検索と表示
        \item 名簿データの整列
    \end{enumerate}
    \item 標準入力からのユーザ入力を通して,データ登録やデータ管理等の操作を可能とすること.
    \item 標準出力には,コマンドの実行結果のみを出力すること.
\end{enumerate}

\subsubsection*{(2) 基本仕様}

\begin{enumerate}
  \setlength{\parskip}{0em} \setlength{\itemsep}{0.25em}  % <-- この一行は項目間の調整
    \item 名簿データは,コンマ区切りの文字列(\textbf{CSV入力}と呼ぶ)で表されるものとし,
          図\ref{fig:csvdata}に示したようなテキストデータを処理できるようにする.%
    \item コマンドは,\%で始まる文字列(\textbf{コマンド入力}と呼ぶ)とし,表\ref{tab:commands}にあげたコマンドをすべて実装する
    \item 1つの名簿データは,C言語の構造体 (\texttt{struct}) を用いて,構造を持ったデータとしてプログラム中に定義し,使用する
    \item 全名簿データは,“何らかのデータ構造”を用いて,メモリ中に保持できるようにする.
    \item コマンドの実行結果以外の出力は,標準エラー出力に出力する.
\end{enumerate}

\subsubsection*{(3) 追加仕様}

\begin{enumerate}
    \item 名簿データの各項目には,コンマは含まれないものとする.
    \item プログラムの入力は,常に半角文字や制御文字のみで構成されてあり,全角文字(例えば,日本語の漢字)は含まれないものとする.
    \item コマンドの主要な機能は,\%に続く1文字で区別されるものとする.
    \item “何らかのデータ構造”は,“固定長の配列”とする.
    \item \verb|int|型は,32 bit(4 bytes)の整数型を扱える変数とする.
\end{enumerate}

%%%%%%%%%%%%%%%%%%%%%%%%%%%%%%%%%%%%%%%%%%%%%%%%%%%%%%%%%%%%%%%%%%%%%%
\section{プログラムの説明}\label{sec:capp}
%%%%%%%%%%%%%%%%%%%%%%%%%%%%%%%%%%%%%%%%%%%%%%%%%%%%%%%%%%%%%%%%%%%%%%

プログラムリストは\ref{sec:makep}章に添付している.最終的なプログラムは全部で627行からなる.
以下では,\ref{sec:abstract}章の課題ごとに,プログラムの主な構造について説明する.
なお,特筆のない限り説明は最新のソースコード(\ref{funclast}節)に基づく.また,多くの関数において例外対策やユーザフレンドドリなメッセージが実装されているが,それらには\ref{advancedfunction}章で触れるため,本章での説明は最小限にとどめる.

%--------------------------------------------------------------------%
\subsection{文字列操作の基礎:\texttt{subst}関数と\texttt{split}関数の実装}
%--------------------------------------------------------------------%
% Tips:
% - subsectionなどでは verbが使えない.texttt を使うとよい.
% - 最初の段落で,この節で説明する関数等の名前や場所などを説明する.
% - 以降の段落で,最初の段落で示した,各関数等に関する詳細を説明する.

まず,汎用的な文字列操作関数として,
\verb|subst()|関数を49--63行目で宣言し,
\verb|split()|関数を65--81行目で宣言している.
また,これらの関数で利用するために,\verb|<stdio.h>|というヘッダファイルをインクルードする.

\verb|subst(str, C1, C2)|関数は,\verb|str|が指す文字列中の,文字\verb|C1|を文字\verb|C2|に置き換える.
プログラム中では,\verb|get_line()|関数内の\verb|fgets()|関数で文字列の入力を受けるとき,末尾に付く改行文字を\verb|NULL|文字で置き換えるために使用している.呼び出し元には,文字を置き換えた回数を戻り値として返す.

\verb|split(str, ret[], sep, max)|関数は,他関数から渡された文字列中に文字変数\verb|sep|の文字に一致する文字があった場合,該当文字を\verb|NULL|文字で置き換え,該当文字の次の文字が格納されているメモリのアドレスを\verb|ret[]|に書き込む.なお,\verb|ret[0]|に格納されるアドレスの値は,\verb|split()|関数が呼び出された際の\verb|str|の値である.以降,\verb|sep|の文字に一致する文字があった場合,\verb|ret[]|の添字を1ずつ増やしながらアドレスを格納していく.呼び出し元には,アドレスが格納されている\verb|ret[]|の内,添字が最も大きいものの添字に1を加えたものを戻り値として返す.

%--------------------------------------------------------------------%
\subsection{構造体や配列を用いた名簿データの定義}
%--------------------------------------------------------------------%

本名簿管理プログラムでは,構造体の配列を名簿データとして扱う.
10--15行目で,\verb|date|構造体を定義し,
17--24行目で,\verb|profile|構造体を定義している.
この\verb|profile|型の変数1つが,名簿データ1つに相当する.
そして,46行目の\verb|profile_data_store|変数で,全名簿データを管理し,
47行目の\verb|int|型変数\verb|profile_data_nitems|で,名簿データの個数を管理する.
\verb|date|構造体の定義にあたっては,年,月,日に分けて情報を管理できるよう,3つの\verb|int|型変数を用意している.
\verb|profile|構造体は,その要素に\verb|date|構造体を含む.

\verb|profile|構造体の各要素について説明する.
まず,IDを格納するための\verb|int|型変数\verb|id|である.これは\verb|int|型変数の最小値\verb|-2147483648|と最大値\verb|2147483647|の間で整数値を格納できる.
次に,氏名を格納するための\verb|char|型の配列\verb|name|である.これは\verb|NULL|文字を含めて最大70文字の文字列を格納できる.
そして,\verb|date|型の変数\verb|birthday|である.これは前述したように年,月,日の3つの\verb|int|型のメンバからなる変数である.
次に,住所を格納するための\verb|char|型の配列\verb|address|である.文字列の配列\verb|name|同様,\verb|NULL|文字を含めて最大70文字の文字列を格納できる.
最後に,備考を格納するための\verb|char|型のポインタ\verb|biko|である.文字列の先頭アドレスを格納する.

%--------------------------------------------------------------------%
\subsection{標準入力の取得と構文解析}
%--------------------------------------------------------------------%

標準入力を取得するための\verb|get_line()|関数は83--91行目で宣言している.構文解析のための\verb|parse_line()|関数は,93--113行目で宣言している.標準入出力のため,\verb|<stdio.h>|というヘッダファイルをインクルードする.

\verb|get_line(line)|関数は,標準入力\verb|stdin|を\verb|fgets()|関数で取得し,1024文字以上を越えた場合は,次の行として処理を行うことでバッファオーバーランを防止している.標準入力が\verb|NULL|のときは,呼び出し元に戻り値$0$を返して関数は終了する.制御文字\verb|ESC|を1文字目に入力した場合は,コマンド\verb|%Q|を引数\verb|r|つきで呼び出した動作(デバッグ用の機能)を行う.それ以外の場合,\verb|subst()|関数で末尾の改行文字を\verb|NULL|文字に置き換えた後,呼び出し元に戻り値$1$を返す.

\verb|parse_line(line)|関数は,他関数からの文字列配列をポインタ\verb|line|で取得し,入力内容が特定のコマンドとその引数であるか,名簿データを入力した文字列であるかを判定し,それぞれ\verb|exec_command()|関数を呼び出すか,\verb|new_profile()|関数を呼び出す.\verb|get_line()|関数で入力された入力文字列の1文字目が\%である場合,\verb|split()|関数を呼び出し,\verb|line|の2文字目のアドレス,ポインタの配列\verb|ret|,区切り文字である半角スペース,最大区切り数$2$を渡す.そして,\verb|exec_command()|関数を呼び出し,\verb|ret[0]|,\verb|ret[1]|を引数として渡す.
1文字目が\verb|%|でない場合は,名簿データが既に10000件登録されていないことを確認し,名簿データの新規登録を行う\verb|new_profile()|関数を呼び出す.既に10000件登録されている場合は登録を中止し,エラーメッセージを表示する.登録を中止した件数は\verb|int|型の変数\verb|i|でカウントする.
%--------------------------------------------------------------------%
\subsection{\texttt{CSV}データ登録処理の実装}
%--------------------------------------------------------------------%

\verb|CSV|データ登録処理を行う\verb|new_profile(profile_p, line)|関数を,198--236行で宣言している.

\verb|new_profile()|関数で,\verb|profile|型のグローバル変数\verb|profile_data_store[10000]|に\verb|CSV|データの入力情報を登録する.何番目のデータとして入力情報の登録を行うかは,\verb|int|型のグローバル変数\verb|profile_data_nitems|によって指定する.\verb|profile_data_nitems|は\verb|new_profile()|関数を呼び出す際にインクリメントされるため,重複なくデータの登録が行われる.まず,\verb|input_format_check()|関数により,文字列\verb|line|をカンマ,セミコロン,タブのどの文字で区切るかを検討する.いずれかの文字が4回\verb|line|に含まれていれば,カンマ,セミコロン,タブの優先度で区切りを行う用意をする.戻り値は区切る文字の文字コードとなっている.いずれの文字でも区切れないと見込まれる場合,エラーメッセージを表示し,処理を中止する.処理を中止した場合,変数\verb|profile_data_store|への代入処理を行わないため,\verb|data_profile_nitems|のみが,\verb|new_profile()|関数の呼び出し時にインクリメントされた状態になる.このままだと,要素が入らない状態になってしまうため,\verb|new_profile()|関数を終了する前に,\verb|profile_data_nitems|をデクリメントする.それとともに登録を中止した件数を\verb|int|型の変数\verb|i|でカウントする.次に,\verb|split()|関数で1行分の入力をカンマを基準にID,氏名,誕生日,住所,備考に分け,ポインタ配列\verb|ret[]|にそれぞれの文字列要素の先頭アドレスを格納する.続いてIDが\verb|int|型の値に\verb|atoi()|関数で正常に変換できることを\verb|int_value_check()|関数で確認し,かつ変換後に負の値を取らないことを確認する.誕生日は\verb|split()|関数でハイフンを基準に年,月,日の要素に分け,ポインタ配列\verb|ret2[]|にそれぞれの要素を格納する.こちらは\verb|split()|関数の戻り値を用いて,年,月,日分割できていることを確認し,分けられていない場合は処理を中止する.誕生日の年,月,日それぞれが\verb|atoi()|関数で正常に変換できるか,\verb|int_value_check()|関数で確認を行う.IDとは異なり,こちらは負の値にならないかどうかを確認する必要はない.区切り文字である‘\verb|-|’は,マイナスとハイフンで共通の文字であるからである.続いて,入力された誕生日の情報が実在する日付けかどうか,\verb|day_format_check()|関数を用いて確認を行う.

処理が中止されなかった場合,次の代入処理を行う.
まず,IDの代入を行う.ただし,このIDに対応する変数\verb|profile_data_store|のメンバは\verb|profile|構造体で,\verb|int|型の値として宣言されている.もともとの入力は文字列であるため,これをそのまま代入することはできない.例えば,入力されたID情報が\verb|437|だったとしても,それは文字‘\verb|4|’,‘\verb|3|’,‘\verb|7|’のことであり,整数値\verb|437|のことではない.この文字列\verb|437|を\verb|int|型の変数に代入するため,\verb|atoi()|関数を用いる.\verb|atoi()|関数は,文字列で表現された数値を\verb|int|型の整数値に変換するものである.変換不能な文字列の場合,結果は\verb|0|となる\cite{www:label6}.次に氏名の情報,これに対応するメンバは\verb|name|であるので,文字列をそのまま代入することができる.ただし,C言語において,文字列を\verb|=|で結んで代入することはできないため,\verb|strncpy()|関数を用いる.今回は,代入する文字列の最大長が70と予め決まっているため,\verb|strncpy()|関数を用いて,70文字を越えた文字列の代入を阻止している\cite{www:label5}.続いて,誕生日を\verb|date|構造体のメンバである\verb|y|,\verb|m|,\verb|d|に分けて代入する.\verb|ret2[0]|,\verb|ret2[1]|,\verb|ret2[2]|がそれぞれ対応する値になっているが,これもIDの場合と同様で文字列であるので,\verb|atoi()|関数を用いて,\verb|int|型の整数値に変換してから代入する.住所情報の代入の処理は,氏名情報の代入処理と同じ処理を行うため,説明を省略する.最後に備考情報の登録であるが,備考情報には文字数の制約が無いため,何文字であっても処理が行えなければならない.予め備考の文字数を\verb|strlen()|関数で取得し,int型変数\verb|MAX_BIKO|に格納し,\verb|NULL|文字分の$1$を足しておく\cite{www:label8}.次に,\verb|malloc()|関数にて必要分のメモリを確保する\cite{www:label9}.\verb|malloc()|関数の戻り値は,アドレスであるが,アドレスは4バイト長であるため\verb|int|型なのか\verb|char|型なのかといったことが分からない.そのため,キャスト演算子を用いて\verb|char|型の値を指すアドレスだと明示的に指定する\cite{www:label10}.最後に,\verb|strncpy()|関数で,先程確保したメモリに文字列をコピーしていく.すべての代入処理を終えたあと,\verb|new_profile()|関数は終了する.\verb|void|型の関数であるため,戻り値はない.

%--------------------------------------------------------------------%
\subsection{コマンド中継処理の実装}
%--------------------------------------------------------------------%

コマンド中継処理を行う\verb|exec_command(cmd, param)|関数を,454--531行で宣言している.
なお,使用可能なコマンドは表\ref{tab:commands}に記載している.

この関数は,\verb|parse_line()|関数からコマンドの種類を決定する1-2文字の文字列の先頭アドレスを格納する\verb|cmd|とコマンドの引数とする文字列の先頭アドレスを格納する\verb|param|を受け取る.\verb|seitch|文で\verb|cmd|の値によって,呼び出す関数の一文字目を選択する.存在しないコマンドを呼び出そうとした場合,\verb|default|の項目が実行され,エラーメッセージを出力する.‘\verb|W|’から始まるコマンド群である‘\verb|W|’,“\verb|WS|”,“\verb|WT|”コマンドに関しては,別途strcmp()関数を用いて判定を行う.

\verb|%Q|コマンドに対応する\verb|cmd_quit()|関数を,168--195行目で宣言している.\verb|%Q|コマンドは,\verb|exec_command()|関数の\verb|switch|文中で\verb|cmd_quit()|関数を呼び出し,プログラムを終了する役割を持つ.\verb|cmd_quit()|関数は,終了するかの確認メッセージを出力し,‘\verb|y|’か‘\verb|n|’の入力を求める.引数\verb|r|を入力していれば,確認メッセージは表示しない.正常終了を示す\verb|0|をシェルに返すため,\verb|exit(0)|として,\verb|exit()|関数を呼び出す.\verb|exit()|関数を使用するために,\verb|stdlib.h|をインクルードする.\verb|void|型の関数であるため,戻り値はない.

\verb|%C|コマンドに対応する\verb|cmd_check()|関数を,197--200行目で宣言している.\verb|%C|コマンドは,\verb|exec_command()|関数の\verb|switch|文中で\verb|cmd_check()|関数を呼び出す.\verb|cmd_check()|関数に引数はなく,プログラム中に読み込まれている名簿データの総数を表示する関数である.\verb|int|型のグローバル変数である\verb|profile_data_nitems|に,名簿データの総数を格納しているので,この値を表示する.\verb|void|型の関数であるため,戻り値はない.


%========================= EXAMPLE CSV ================================%
\begin{figure}[b]
\centering % この1行はbegin~endの中を中央寄せにする,というコマンド
% Verbatim environment
% プリアンブルで \usepackage{fancyvrb} が必要.
%   - frame         枠で囲う.single なら,四角で囲われる
%   - xleftmargin   枠の左の余白
%   - xrightmargin  枠の右の余白
%   - gobble        先頭n文字を無視.インデントを無視したい場合に利用する.
%   - fontsize      フォントサイズ指定
%   - numbers       行番号を表示.left なら左に表示.
%   - firstnumber   最初の行番号を指定
\begin{Verbatim}[frame=single, xleftmargin=5mm, xrightmargin=5mm, gobble=4,
                 fontsize=\small, numbers=left, firstnumber=1]
    5100046,The Bridge,1845-11-2,14 Seafield Road Longman Inverness,SEN Unit 2.0 Open
    5100127,Bower Primary School,1908-1-19,Bowermadden Bower Caithness,01955 641225 ...
    5100224,Canisbay Primary School,1928-7-5,Canisbay Wick,01955 611337 Primary 56 3...
    5100321,Castletown Primary School,1913-11-4,Castletown Thurso,01847 821256 01847...
\end{Verbatim}
    \caption{名簿データのCSV入力形式の例.1行におさまらないデータは...で省略した.}
    \label{fig:csvdata}
\end{figure}
%========================= EXAMPLE CSV ================================%

%========================= COMMAND LIST ================================%
\begin{table}[b]  %表の位置は原則として t または b.h や H は使わない.
\centering % この1行はbegin~endの中を中央寄せにする,というコマンド
    \caption{実装するコマンド}
    \label{tab:commands}
    \begin{tabular}{|l|ll|l|}
    \hline
    \textbf{コマンド} & \textbf{意味} &        & \textbf{備考} \\
    \hline\hline
    \verb|%Q|   & 終了                & (Quit)  & \\
    \hline
    \verb|%C|	&登録内容のチェック	& (Check) &1行目に登録数を必ず表示\\
    \hline
    \verb|%P n| & 先頭から$n$件表示   & (Print) & $n$が$0$ $\to$ 全件表示,        \\
                &                     &         & $n$が負 $\to$ 後ろから$-n$件表示 \\
    \hline
    \verb|%R file| & \verb|file|から読込み & (Read) & \\
    \hline
    \verb|%W file| & \verb|file|への書出し & (Write) & CSV形式で書出し \\
    \hline
    \verb|%WS file| & \verb|file|への書出し & (Write) & SCSV形式で書出し \\
    \hline
    \verb|%WT file| & \verb|file|への書出し & (Write) & TSV形式で書出し \\
    \hline
    \verb|%W word| & 検索結果を表示 & (Find) & \verb|%P|と同じ形式で表示 \\
    \hline
    \verb|%S n| & \verb|CSV|の$n$番目の項目で整列 & (Sort) & 表示はしない \\
    \hline
    \end{tabular}
\end{table}
%========================= COMMAND LIST ================================%


%--------------------------------------------------------------------%
\subsection{コマンドの実装:\texttt{\%P}コマンド}
%--------------------------------------------------------------------%

\verb|%P|コマンドに対応する\verb|cmd_print()|関数を,202--244行目で宣言している.\verb|%P|コマンドは,引数の条件に従って名簿データの中身を表示する関数である\verb|cmd_print(*param)|関数を呼び出す.この関数は,\verb|exec_command()|関数経由で,\verb|parse_line()|関数からパラメータ部の先頭アドレスを受け取る.この時点では,引数として入力された数値は文字列の状態なので,\verb|atoi()|関数を用いて,\verb|int|型の整数値に変換して,\verb|int|型の変数\verb|a|に格納する.このとき,\verb|a|の絶対値が\verb|data_profile_nitems|以上か$0$のとき,\verb|a|に\verb|data_profile_nitems|を代入する.これにより,まだ名簿データが格納されていない部分を表示するのを防止している.それ以外の時はそのまま名簿データの表示処理に進む.\verb|a|が正のとき,先頭から\verb|a|件の名簿データを昇順で表示する.\verb|a|が負のとき,後ろから\verb|a|件を昇順で表示する.グローバル変数\verb|data_profile_store[]|の添字に,\verb|for|ループで変化する\verb|int|型変数\verb|i|を用いることで表示する名簿データを指定する.\verb|void|型の関数であるため,戻り値はない.

%--------------------------------------------------------------------%
\subsection{\texttt{\%R}コマンドと\texttt{\%W}コマンド}
%--------------------------------------------------------------------%

\verb|%R|コマンドに対応する\verb|cmd_read(param)|関数を,246--269行目で宣言している.\verb|%R|コマンドは,実行ファイルと同ディレクトリ内にある,\verb|param|で指定された文字列と同じ名前のファイルを開き,その内容を\verb|get_line()|で入力処理する関数である.まず,コマンドの実行に必要な引数が存在するかを確認し,なければエラーメッセージを表示して処理を中止する.また,指定されたファイルが存在せず,開けない場合もエラーメッセージの表示を行い,処理を中止する.次に,\verb|get_line()|関数で1行ずつファイルの内容を読んでいく.引数としてファイルポインタ\verb|fp|と読み込んだ1行分の文字列\verb|char|型の配列\verb|LINE|を渡す.次に\verb|parse_line()|関数で1行分の入力を解析し,対応する関数の呼び出しを行う.すべての文字列を読み込んだら,ファイルはもう使用しないため,ファイルポインタを閉じる.\verb|void|型の関数であるため,戻り値はない.

\verb|%W|コマンドに対応する\verb|cmd_write(param, sep)|関数を,271--299行目で宣言している.\verb|%W|コマンドは,実行ファイルがあるディレクトリ内に\verb|param|で指定された文字列と同じ名前のファイルを作成し,名簿データを\verb|sep|で指定された文字で区切りながら出力する.\verb|%W|コマンドを実行した場合,\verb|sep|の文字はカンマになる.まず,コマンドの実行に必要な引数が存在するかを確認し,なければエラーメッセージを表示して処理を中止する.また,指定されたファイルが読み取り専用であったり,開けなかったりする場合もエラーメッセージの表示を行い,処理を中止する.正常にファイルが開けた場合,

%--------------------------------------------------------------------%
\subsection{\texttt{\%F}コマンド}
%--------------------------------------------------------------------%

\verb|%F|コマンドに対応する\verb|cmd_find(param)|関数を,301--338行目で宣言している.

%--------------------------------------------------------------------%
\subsection{\texttt{\%S}コマンド}
%--------------------------------------------------------------------%

\verb|%S|コマンドに対応する\verb|cmd_sort(param)|関数を,340--420行目で宣言している.

%--------------------------------------------------------------------%
\subsection{独自コマンド}
%--------------------------------------------------------------------%

独自コマンドとして,\verb|%WS|コマンドと\verb|%WT|コマンドを実装した.これらのコマンドは,\verb|cmd_write()|関数の引数である\verb|char|型の値\verb|sep|に代入する文字をセミコロンや水平TABに変更して実行するコマンドである.名簿や住所録などのデータは,最も一般的であるカンマ区切りのCSVファイル以外に,セミコロン区切りであるSCSVファイル,水平TAB区切りであるTSVファイルなども使われている.このコマンドを用いることにより,セミコロン区切りや水平TAB区切りにしたファイルに出力できる.

%%%%%%%%%%%%%%%%%%%%%%%%%%%%%%%%%%%%%%%%%%%%%%%%%%%%%%%%%%%%%%%%%%%%%%
\section{プログラムの使用法と実行結果}\label{sec:howresult}
%%%%%%%%%%%%%%%%%%%%%%%%%%%%%%%%%%%%%%%%%%%%%%%%%%%%%%%%%%%%%%%%%%%%%%

本プログラムは名簿データを管理するためのプログラムである.
CSV形式のデータと \% で始まるコマンドを標準入力から受け付け,
処理結果を標準出力に出力する.
入力形式の詳細については,\ref{sec:plan}節で説明した.

% 実行結果を得るために利用した,実行環境のOSを必ず書くこと.
プログラムは,CentOS 7.8.2003(Core) で動作を確認しているが,
一般的な UNIX で動作することを意図している.
なお,以降の実行例における,行頭の\verb|$|記号は,
CentOS 7.8.2003(Core) におけるターミナルのプロンプトである.

まず,\verb|gcc|でコンパイルすることで,プログラムの実行ファイルを生成する.
ここで,\verb|-Wall|とは通常は疑わしいものとみなされることのない構文に関して警告を出力するためのオプションであり,
\verb|-o|とは出力ファイル名を指定するオプションである.
これらのオプションをつけることで,疑わしい構文を発見し,任意の出力ファイル名を指定することができる.

{\fontsize{10pt}{11pt} \selectfont
 \begin{verbatim}
   $ gcc program1.c -o program1 -Wall
 \end{verbatim}
}
%% 注:行送りの変更は"指定箇所を含む段落”に効果があらわれる.
%%     fontsizeコマンドを用いて,行送りを変える場合は,
%%     その {...} の前後に空白行を入れ,段落を変えるようにすること.
%%     なお,行先頭がコメントから始まる行は空白行とは扱われない.

次に,プログラムを実行する.
以下の実行例は,プログラム実行中の動作例を模擬するため,
任意の\verb|csv|ファイルを標準入力のリダイレクションにより与えることで,実行する例を示している\cite{www:label1}.
\verb|csv|ファイルの内容は,図\ref{fig:csvdata}のような体裁である.
通常の利用においては,キーボードから文字列を入力してもよい.

{\fontsize{10pt}{11pt} \selectfont
 \begin{verbatim}
   $ ./program1 < csvdata.csv
 \end{verbatim}
}

以上のようにして,ファイルを標準入力のリダイレクションで与える.

まず,\verb|subst|関数の実行結果について説明する.
\ref{func1}節のプログラムに対して,次の内容の\verb|input.txt|をリダイレクションで与えた場合,

{\fontsize{10pt}{11pt} \selectfont
 \begin{verbatim}
Apple
p
a
 \end{verbatim}
}

\noindent % noindentとはここでは段落を変えない(一字下げをしない)というコマンド.
以下のような出力が得られる.

{\fontsize{10pt}{11pt} \selectfont
 \begin{verbatim}
str:Aaale
count:2
 \end{verbatim}
}

入力データについて説明する.
最初の1行は,基準の文字列である.
2行目に基準の文字列で置き換えたい1文字を入力する.3行目には,置き換え語の1文字を入力する.
出力結果では,2行目で入力した1文字が,3行目で入力した1文字が置き換わっている.\verb|count|の項目は,置き換えた文字数である.

次に,\verb|split()|,\verb|get_line()|関数の実行結果について説明する.
\ref{func2}節のプログラムに対して,次の内容の\verb|input.txt|をリダイレクションで与えた場合,

{\fontsize{10pt}{11pt} \selectfont
 \begin{verbatim}
Microsoft,Windows,7,Ultimate,7601,24545
Apple,iMac,Late,2013,A1418
 \end{verbatim}
}

\noindent % noindentとはここでは段落を変えない(一字下げをしない)というコマンド.
以下のような出力が得られる.

{\fontsize{10pt}{11pt} \selectfont
 \begin{verbatim}
line number:1
input:"Microsoft,Windows,7,Ultimate,7601,24545"
split[0]:"Microsoft"
split[1]:"Windows"
split[2]:"7"
split[3]:"Ultimate"
split[4]:"7601"
split[5]:"24545"


line number:2
input:"Apple,iMac,Late,2013,A1418"
split[0]:"Apple"
split[1]:"iMac"
split[2]:"Late"
split[3]:"2013"
split[4]:"A1418"
 \end{verbatim}
}

入力データについて説明する.
出力ブロックの最初の1行は,\verb|CSV|ファイルを想定した,カンマで区切られた文字列である.
出力結果では,何行目の入力かを示す\verb|line number|の項目と,実際の入力内容を確認で表示し,3行目以降にカンマで区切られた文字列が出力される.\verb|get_line()|関数は入力が\verb|NULL|になると入力処理を終了するように$0$を戻り値として返すので,出力結果は\verb|line number|が$2$の項目までとなっている.

次に,\verb|parse_line()|,\verb|new_profile()|,\verb|cmd_check()|,\verb|cmd_print()|,\verb|cmd_quit()|関数の実行結果について説明する.
\ref{func5}節のプログラムに対して,次の内容の\verb|input.txt|をリダイレクションで与えた場合,

{\fontsize{10pt}{11pt} \selectfont
 \begin{verbatim}
5100046,The Bridge,1845-11-2,14 Seafield Road Longman Inverness,SEN Unit 2.0 Open
5100127,Bower Primary School,1908-1-19,Bowermadden Bower Caithness,01955 641225 Primary
 25 2.6 Open
5100224,Canisbay Primary School,1928-7-5,Canisbay Wick,01955 611337 Primary 56 3.5 Open
%C
%P 2
%P -2
%P 5
%P 0
%Q
 \end{verbatim}
}

\noindent % noindentとはここでは段落を変えない(一字下げをしない)というコマンド.
以下のような出力が得られる.

{\fontsize{10pt}{11pt} \selectfont
 \begin{verbatim}
3 profile(s)
Id    : 5100046
Name  : The Bridge
Birth : 1845-11-02
Addr. : 14 Seafield Road Longman Inverness
Comm. : SEN Unit 2.0 Open

Id    : 5100127
Name  : Bower Primary School
Birth : 1908-01-19
Addr. : Bowermadden Bower Caithness
Comm. : 01955 641225 Primary 25 2.6 Open

Id    : 5100127
Name  : Bower Primary School
Birth : 1908-01-19
Addr. : Bowermadden Bower Caithness
Comm. : 01955 641225 Primary 25 2.6 Open

Id    : 5100224
Name  : Canisbay Primary School
Birth : 1928-07-05
Addr. : Canisbay Wick
Comm. : 01955 611337 Primary 56 3.5 Open

Id    : 5100046
Name  : The Bridge
Birth : 1845-11-02
Addr. : 14 Seafield Road Longman Inverness
Comm. : SEN Unit 2.0 Open

Id    : 5100127
Name  : Bower Primary School
Birth : 1908-01-19
Addr. : Bowermadden Bower Caithness
Comm. : 01955 641225 Primary 25 2.6 Open

Id    : 5100224
Name  : Canisbay Primary School
Birth : 1928-07-05
Addr. : Canisbay Wick
Comm. : 01955 611337 Primary 56 3.5 Open

Id    : 5100046
Name  : The Bridge
Birth : 1845-11-02
Addr. : 14 Seafield Road Longman Inverness
Comm. : SEN Unit 2.0 Open

Id    : 5100127
Name  : Bower Primary School
Birth : 1908-01-19
Addr. : Bowermadden Bower Caithness
Comm. : 01955 641225 Primary 25 2.6 Open

Id    : 5100224
Name  : Canisbay Primary School
Birth : 1928-07-05
Addr. : Canisbay Wick
Comm. : 01955 611337 Primary 56 3.5 Open

 \end{verbatim}
}

入力データについて説明する.
出力ブロックの最初の3行は,\verb|CSV|ファイルを想定した,カンマで区切られた文字列である.
以降の行では,\verb|%|から始まるコマンドの呼び出しを行っている.
出力結果では,最初の1行に名簿データの登録数を出力する\verb|cmd_check()|関数による\verb|3 profile(s)|という内容が出力されている.それ以降のブロックでは,\verb|%P 2|という引数を含む\verb|cmd_print()|関数の呼び出しで,名簿データの初めから2件が表示され,次に\verb|%P -2|により,名簿データの後ろから2件を降順に表示している.次の\verb|%P 5|は名簿データが3件しか登録されていないため,\verb|%P 3|に読み替えて処理が行われる.引数が無い場合や$0$の場合も同様となる.最後に\verb|%Q|により,プログラムを終了している.

次に,\verb|cmd_read()|\verb|cmd_write()|,\verb|cmd_find()|,\verb|cmd_quit("r")|関数の実行結果について説明する.
\ref{funclast}節のプログラムに次の内容の\verb|input.txt|をリダイレクションで与えた場合,

{\fontsize{10pt}{11pt} \selectfont
 \begin{verbatim}
%R input.csv
%C
%F 5100127
%W output.csv
%Q r
 \end{verbatim}
}

\noindent
なお,\verb|input.csv|の中身は図\ref{input.csv}に示す.
\begin{figure}[b]
{\fontsize{10pt}{11pt} \selectfont
 \begin{verbatim}
5100046,The Bridge,1845-11-2,14 Seafield Road Longman Inverness,SEN Unit 2.0 Open
5100127,Bower Primary School,1908-1-19,Bowermadden Bower Caithness,01955 641225 Primary
 25 2.6 Open
5100224,Canisbay Primary School,1928-7-5,Canisbay Wick,01955 611337 Primary 56 3.5 Open
 \end{verbatim}
}
\caption{input.csv}
\label{input.csv}
\end{figure}

\noindent % noindentとはここでは段落を変えない(一字下げをしない)というコマンド.
以下のような出力が得られる.

{\fontsize{10pt}{11pt} \selectfont
 \begin{verbatim}
3 profile(s)
Id    : 5100127
Name  : Bower Primary School
Birth : 1908-01-19
Addr. : Bowermadden Bower Caithness
Comm. : 01955 641225 Primary 25 2.6 Open

正常終了.
 \end{verbatim}
}

入力データについて説明する.
まず,\verb|%R|コマンドにより,ファイル\verb|input.csv|を読み込む.その次に\verb|%C|コマンドにより,名簿データの登録数を表示する.次に,\verb|%F|コマンドにより,\verb|5100127|に一致する要素を持つ名簿情報を表示する.そして,\verb|%W|コマンドにより,ファイル\verb|output.csv|に名簿データを書き込む.最後に,\verb|%Q r|により,確認メッセージなしでプログラムを終了している.
出力結果では,最初の1行に名簿データの登録数を出力する\verb|cmd_check()|関数による\verb|3 profile(s)|という内容が出力されている.これにより,\verb|%R|コマンドにより,正常に名簿データが登録されていることが分かる.7行目には,\verb|%Q r|により呼び出された\verb|cmd_quit()|関数のメッセージが出力されている.
この操作を実行した後,シェル上で\verb|ls|コマンドを使ってファイルを確認すると次のような状態になっている.

{\fontsize{10pt}{11pt} \selectfont
 \begin{verbatim}
[user@localhost 10]$ ls
input.csv  input.txt  output.csv  program.out
[user@localhost 10]$ diff -q -s input.csv output.csv
ファイル input.csv と output.csv は同一です
 \end{verbatim}
}

diffコマンドを用いて,元の\verb|input.csv|と\verb|output.csv|を比較すると,全く同じ内容になっていることが分かるので,\verb|%W|コマンドが正しく動作していることが分かる.

最後に,\verb|cmd_sort()|関数の実行結果について説明する.
\ref{funclast}節のプログラムに次の内容の\verb|input.txt|をリダイレクションで与えた場合,

{\fontsize{10pt}{11pt} \selectfont
 \begin{verbatim}
%R input.csv
%S 1
%P
%S 2
%P
%S 3
%P
%S 4
%P
%S 5
%P
%Q r
 \end{verbatim}
}

\noindent % noindentとはここでは段落を変えない(一字下げをしない)というコマンド.
以下のような出力が得られる.

{\fontsize{10pt}{11pt} \selectfont
 \begin{verbatim}
Id    : 5100046
Name  : The Bridge
Birth : 1845-11-02
Addr. : 14 Seafield Road Longman Inverness
Comm. : SEN Unit 2.0 Open

Id    : 5100127
Name  : Bower Primary School
Birth : 1908-01-19
Addr. : Bowermadden Bower Caithness
Comm. : 01955 641225 Primary 25 2.6 Open

Id    : 5100224
Name  : Canisbay Primary School
Birth : 1928-07-05
Addr. : Canisbay Wick
Comm. : 01955 611337 Primary 56 3.5 Open

Id    : 5100127
Name  : Bower Primary School
Birth : 1908-01-19
Addr. : Bowermadden Bower Caithness
Comm. : 01955 641225 Primary 25 2.6 Open

Id    : 5100224
Name  : Canisbay Primary School
Birth : 1928-07-05
Addr. : Canisbay Wick
Comm. : 01955 611337 Primary 56 3.5 Open

Id    : 5100046
Name  : The Bridge
Birth : 1845-11-02
Addr. : 14 Seafield Road Longman Inverness
Comm. : SEN Unit 2.0 Open

Id    : 5100046
Name  : The Bridge
Birth : 1845-11-02
Addr. : 14 Seafield Road Longman Inverness
Comm. : SEN Unit 2.0 Open

Id    : 5100127
Name  : Bower Primary School
Birth : 1908-01-19
Addr. : Bowermadden Bower Caithness
Comm. : 01955 641225 Primary 25 2.6 Open

Id    : 5100224
Name  : Canisbay Primary School
Birth : 1928-07-05
Addr. : Canisbay Wick
Comm. : 01955 611337 Primary 56 3.5 Open

Id    : 5100046
Name  : The Bridge
Birth : 1845-11-02
Addr. : 14 Seafield Road Longman Inverness
Comm. : SEN Unit 2.0 Open

Id    : 5100127
Name  : Bower Primary School
Birth : 1908-01-19
Addr. : Bowermadden Bower Caithness
Comm. : 01955 641225 Primary 25 2.6 Open

Id    : 5100224
Name  : Canisbay Primary School
Birth : 1928-07-05
Addr. : Canisbay Wick
Comm. : 01955 611337 Primary 56 3.5 Open

Id    : 5100224
Name  : Canisbay Primary School
Birth : 1928-07-05
Addr. : Canisbay Wick
Comm. : 01955 611337 Primary 56 3.5 Open

Id    : 5100127
Name  : Bower Primary School
Birth : 1908-01-19
Addr. : Bowermadden Bower Caithness
Comm. : 01955 641225 Primary 25 2.6 Open

Id    : 5100046
Name  : The Bridge
Birth : 1845-11-02
Addr. : 14 Seafield Road Longman Inverness
Comm. : SEN Unit 2.0 Open

正常終了.

 \end{verbatim}
}

入力データについて説明する.
まず,\verb|%R|コマンドにより,ファイル\verb|input.csv|を読み込む.その次に\verb|%S|コマンドにより,ID,氏名,誕生日,住所,備考の順に,それぞれ並び変える処理と表示する処理を繰り返す.最後に,\verb|%Q r|により,確認メッセージなしでプログラムを終了している.
出力結果では,18行毎に指定された要素によるソート後の結果を出力している.初めの18行でIDに関する並び替えを行ったもの,その次の18行で氏名に関する並び替えを行ったものというように,表示している.全ての要素はASCIIコードの昇順になるように並び替えが行われる.正しく並び替えが行われていることを確認するために,\verb|%S 1|により並び替えを行ったものを出力したファイル\verb|output.csv|ものと\verb|sort|コマンドにより並び替えを行ったファイル\verb|sort.txt|を比較した.

{\fontsize{10pt}{11pt} \selectfont
 \begin{verbatim}
[user@localhost 10]$ sort input.csv sort.txt
[user@localhost 10]$ diff -q -s output.csv sort.txt
ファイル output.csv と sort.txt は同一です
 \end{verbatim}
}

よって,\verb|%S|コマンドにより,正しく並び替えが行えているこがを確認できた.

%%%%%%%%%%%%%%%%%%%%%%%%%%%%%%%%%%%%%%%%%%%%%%%%%%%%%%%%%%%%%%%%%%%%%%
\section{考察}
%%%%%%%%%%%%%%%%%%%%%%%%%%%%%%%%%%%%%%%%%%%%%%%%%%%%%%%%%%%%%%%%%%%%%%

\ref{sec:capp}章のプログラムの説明,および,\ref{sec:howresult}章の使用法と実行結果から,
演習課題として作成したプログラムが,
\ref{sec:abstract}章で述べた基本要件と基本仕様のいずれも満たしていることを示した.
ここでは,個別の課題のうち,以下の6つの項目について,考察を述べる.
% 注:\ref コマンドを用いて適切に参照すること.

\begin{enumerate}
\setlength{\parskip}{2pt} \setlength{\itemsep}{2pt}
    \item 文字列操作の基礎:\texttt{subst}関数と\texttt{split}関数の実装
    \item 標準入力の取得と構文解析
    \item \texttt{CSV}データ登録処理の実装
    \item \texttt{\%R}コマンドと\texttt{\%W}コマンド
    \item \texttt{\%F}コマンド
    \item \texttt{\%S}コマンド
\end{enumerate}

%--------------------------------------------------------------------%
\subsection{「文字列操作の基礎:\texttt{subst}関数と\texttt{split}関数の実装」に関する考察}
%--------------------------------------------------------------------%

\verb|C|言語では,配列を関数の戻り値とすることはできないので,ポインタを用いて呼び出し元関数の文字列を参照するか,配列専用の構造体を用意する必要がある.例えば,関数の宣言時に構造体名,関数名と書き,構造体のメンバに\verb|char a[70]|などを指定しておくと,70文字までの要素を戻り値として呼び出し元に返すことも可能である(\ref{funcstr}節に実装例を示す.).ただ,二重に代入処理を行うことになるため,扱うデータ量が多くなると効率的ではない.そのため,\verb|subst()|関数が,他関数から文字列の配列の先頭アドレスを受け取ることを想定して,ポインタを使用して他関数内の配列を参照できるようにした.また,入力文字列の途中に\verb|NULL|文字が出現することは標準入力では起こらないため,文字置き換えのループ処理の継続条件に\verb|NULL|文字を用いることで,確実に入力文字列の終端である\verb|NULL|文字手前まで文字の置き換えを行えるようにした.\verb|c1|と\verb|c2|に同じ文字が入力された場合,文字の置き換え処理は実質行われないに等しい.50行目に\verb|break|文を書くことにより,文字の置き換えと置き換えられた文字のカウントを行わず,処理の簡略化及び高速化をしている.
\verb|split()|関数の実装に当たっては,呼び出し元関数で\verb|char|型の二次元配列を作り,そこに文字列情報を複写する方法を考えたが,配列である以上,仮に文字が代入されなくてもメモリを使用してしまうので,メモリを浪費する.メモリを浪費しないために,ポインタと文字列の先頭アドレスのセットで情報を管理することにした.ただ,この方法は,元の入力文字列の情報を破壊するため,場合によっては使用できない.元の入力文字列情報を残す場合は,予め別の変数に文字列をコピーしておく必要がある.

%--------------------------------------------------------------------%
\subsection{「標準入力の取得と構文解析」に関する考察}
%--------------------------------------------------------------------%

\verb|get_line()|関数内の\verb|fgets()|関数で標準入力を取得する際,\verb|%Q|コマンドが未実装の場合,直接入力では\verb|Ctrl+D|で標準入力を\verb|NULL|にすることができるが,デバッグのときの終了が不便である.デバッグをより早く行うために,\verb|ESC|を1文字目に入力することにより,入力待ちを終了できるように81行目に戻り値の条件を追加した(\ref{func4}節のソースコード.\ref{func5}節以降のソースコードでは,\verb|cmd_quit()|関数の呼び出しに機能を変更.).\verb|ESC|は制御文字であり,ファイルリダイレクションによりファイルから入力されることはないため,これを追加した\cite{www:label2}.ファイルリダイレクションから\verb|fgets()|関数により取得された文字列は,終端が改行文字になってしまうため,\verb|subst()|関数を呼び出し,改行文字を\verb|NULL|文字に置き換える処理も含めている.1文字の代入処理であるから,手動で最後のみ\verb|NULL|文字を代入できなくもないが,入力文字列が常に1024文字とは限らないため,改行文字が配列上のどこに来るかは定かではない.そのため,ここでは\verb|subst()|関数を用いる.この処理を行うことで,\verb|subst()|関数のループ処理の継続条件,\verb|split()|関数のループ処理の継続条件や\verb|printf()|関数による文字列の出力や\verb|atoi()|関数の処理などで文字列が扱いやすくなった.\verb|parse_line()|関数では,例外が発生する可能性が多いので,3文字目以降に入力がない場合の対策が必要だと考えた.プログラムの動作を\verb|db-sample|に合わせるため,プログラミング演習1では対策を見送った.また\verb|parse_line()|関数自体が,他関数から文字列の配列の先頭アドレスを受け取り,ポインタ\verb|line|に格納するが,他関数にポインタ\verb|line|をそのまま渡す可能性もあったため,ポインタの値自体をインクリメントしたり,デクリメントしたりすることは避けている.

%--------------------------------------------------------------------%
\subsection{「\texttt{CSV}データ登録処理の実装」に関する考察}
%--------------------------------------------------------------------%

\verb|new_profile()|関数内で,\verb|profile|構造体のメンバである\verb|int|型の変数\verb|id|にID情報を代入する際に,型変換のためキャスト演算子を使用した代入が行えるのではないかと考えたが,キャスト演算子が対象にできるのは単一の値であるため,IDの文字列の一例\verb|437|では,初めの文字である‘\verb|4|’の部分しかキャストできない.これは,2つの配列間において,代入演算子による文字列の代入ができないのと同様の理由が原因である.また,キャストを行うと文字‘\verb|4|’ではなく,文字コードである\verb|52|が代入されてしまうため,値を\verb|-48|するなどの対策も必要になる.また,\verb|atoi()|関数を用いた場合と異なり,‘\verb|A|’(文字コード\verb|65|)と言った本来整数値ではないものの代入を行ってしまう可能性もある.以上より,キャスト演算子ではなく\verb|atoi()|関数を使うべきとの結論に至った.ただし,\verb|atoi()|関数は引数が\verb|NULL|の場合コアダンプが起こるので,引数が\verb|NULL|にならないよう注意する必要がある\cite{www:label7}.\verb|profile|構造体のメンバである氏名,住所への文字列の代入では,70字という字数制限があるので,\verb|strncpy()|関数を用いることで,字数制限を超えない代入が可能となっている.\verb|fgets()|関数同様,\verb|strncpy()|関数の文字数には,\verb|NULL|文字はカウントされていないので,引数は69とした.備考は,文字数が任意であるため,予め配列などで文字列を長を決めてしまうと,メモリを浪費してしまうため,備考情報の文字列長を\verb|strlen()|関数で取得し,それに\verb|NULL|文字分の1を加えた分のメモリを\verb|malloc()|関数で新たに確保し,それの先頭アドレスを構造体のメンバである\verb|*biko|に代入することで,メモリの浪費を防止できる.

%--------------------------------------------------------------------%
\subsection{「\texttt{\%R}コマンドと\texttt{\%W}コマンド」に関する考察}
%--------------------------------------------------------------------%


%--------------------------------------------------------------------%
\subsection{「\texttt{\%F}コマンド」に関する考察}
%--------------------------------------------------------------------%


%--------------------------------------------------------------------%
\subsection{「\texttt{\%S}コマンド」に関する考察}
%--------------------------------------------------------------------%

%%%%%%%%%%%%%%%%%%%%%%%%%%%%%%%%%%%%%%%%%%%%%%%%%%%%%%%%%%%%%%%%%%%%%%
\section{発展課題} \label{advancedfunction}
%%%%%%%%%%%%%%%%%%%%%%%%%%%%%%%%%%%%%%%%%%%%%%%%%%%%%%%%%%%%%%%%%%%%%%

%--------------------------------------------------------------------%
\subsection{エラー処理についての考察}
%--------------------------------------------------------------------%

%--------------------------------------------------------------------%
\subsection{コマンド入力仕様の拡張}
%--------------------------------------------------------------------%

%--------------------------------------------------------------------%
\subsection{既存機能や既存コマンドの拡張}
%--------------------------------------------------------------------%

%%%%%%%%%%%%%%%%%%%%%%%%%%%%%%%%%%%%%%%%%%%%%%%%%%%%%%%%%%%%%%%%%%%%%%
\section{感想}\label{sec:review}
%%%%%%%%%%%%%%%%%%%%%%%%%%%%%%%%%%%%%%%%%%%%%%%%%%%%%%%%%%%%%%%%%%%%%%

毎回の講義でプログラムのソースコードを追記していくとき,どの関数を上から順に書けばいいかを考えていたが,煩雑化してしまうため,関数プロトタイプ宣言をすることで見やすくなり作業がしやすくなった\cite{book:meikai}.今までの自作プログラムでは,関数プロトタイプ宣言が必要になるほど自作関数を用意することがなかったので,実際のプログラミングの経験になった.また,プログラム全体を通して表記のゆれが少なくなるように,多くの自作関数において,ポインタの指し先をずらすために\verb|int|型変数\verb|i|を使用したり,各関数内で似たような役割を持つ変数の変数名を統一したりした.今回の課題プログラムの作成を通して,ポインタへの理解が一層深まり,ポインタのポインタやポインタの配列に関して理解することができたと思う\cite{www:label3,www:label4}.\verb|printf()|関数の書式指定\verb|%s|の引数として,任意のアドレスを代入するが,ポインタの配列\verb|*ret[]|の場合,どのような形で書けばいいのか悩んだ.\verb|printf()|関数の書式指定\verb|%c|の引数では,引数が文字コードなので,ポインタを利用して引数を指定する場合\verb|*|を付ける必要があった.一方書式指定\verb|%s|の場合は,引数がアドレスのため\verb|*|は必要なく,単に\verb|ret[]|と書くので混乱した.アドレスなのか値なのか,今後プログラムを書くときに注意しておきたい.構造体のメンバに他の構造体を含む,構造体の宣言は今回の演習で初めて行ったが,ポインタを用いた構造体のメンバの指定の仕方が,\verb|(profile_p->birthday).y|となり,\verb|(profile_p->birthday)->y|という書き方ができないのが何故なのか疑問点のままだ.プログラム中では\verb|profile_p|は,ポインタであり,ポインタの指し先のメンバとなる\verb|birthday|との関係は,ポインタとその指し先のメンバだが,\verb|birthday|とそのメンバである\verb|y|との関係は,構造体とそのメンバの関係となるので,“\verb|.|”でつなぐことができるのかもしれない.\ref{func2}節のプログラムにおける\verb|split()|関数の作成で,配列の添字に合わせて,戻り値が分けられた個数より1小さくなっていたのを\verb|main()|関数のループの条件で矯正していたが,これが最終的にプログラムの仕様を満たさなくなっていたということがあり,プログラミング演習1の最後で行った基本関数のテストでNGとなった.このため,最終的にソースコードの修正を行った.修正後の最終プログラムは\ref{func6}節に記載している.\verb|split()|関数内で起きたトラブルを他関数側の処理で矯正するのは避けようと思った.

%%%%%%%%%%%%%%%%%%%%%%%%%%%%%%%%%%%%%%%%%%%%%%%%%%%%%%%%%%%%%%%%%%%%%%
\section{作成したプログラム}\label{sec:makep}
%%%%%%%%%%%%%%%%%%%%%%%%%%%%%%%%%%%%%%%%%%%%%%%%%%%%%%%%%%%%%%%%%%%%%%

作成したプログラムリストを以下に添付する.
なお,\ref{sec:abstract}章に示した課題については,
\ref{sec:howresult}章で示したようにすべて正常に動作したことを付記しておく.

% Verbatim environment
% プリアンブルで \usepackage{fancyvrb} が必要.
%   - numbers           行番号を表示.left なら左に表示.
%   - xleftmargin       枠の左の余白.行番号表示用に余白を与えたい.
%   - numbersep         行番号と枠の間隙 (gap).デフォルトは 12 pt.
%   - fontsize          フォントサイズ指定
%   - baselinestretch   行間の大きさを比率で指定.デフォルトは 1.0.

\subsection{プログラミング演習1第1回講義のプログラム}\label{func1}
\verb|subst()|関数を加えたプログラムのソースコード(41行)
\begin{Verbatim}[fontsize=\small, baselinestretch=0.8]
     1	#include <stdio.h>
     2	
     3	int subst(char *str, char c1, char c2)
     4	{
     5	  int i;
     6	  int c = 0;
     7	  for(i = 0; *(str + i) != '\0'; i++)
     8	    {
     9	      if(c1 == c2) break;
    10	      if(*(str + i) == c1)
    11		{
    12		  *(str + i) = c2;
    13		  c++;
    14		}
    15	    }
    16	  return c;
    17	}
    18	
    19	int main(void)
    20	{
    21	  char str[100] = {0};
    22	  char c1 = 0;
    23	  char c2 = 0;
    24	  char dummy;
    25	  int c = 0; 
    26	
    27	  printf("Input str.\n");
    28	  scanf("%s", str);
    29	  printf("Input c1.\n");
    30	  scanf("%c", &dummy);
    31	  scanf("%c", &c1);
    32	  printf("Input c2.\n");
    33	  scanf("%c", &dummy);
    34	  scanf("%c", &c2);
    35	
    36	  c = subst(str, c1, c2);
    37	  
    38	  printf("\nstr:%s\ncount:%d\n", str, c);
    39	
    40	  return 0;
    41	}
\end{Verbatim}

\subsection{プログラミング演習1第2回講義のプログラム}\label{func2}
\verb|subst()|,\verb|get_line()|関数を加えたプログラムのソースコード(70行)
\begin{Verbatim}[fontsize=\small, baselinestretch=0.8]
     1	#include <stdio.h>
     2	
     3	#define ESC 27                                             /*文字列ESCをESCのASCII
コードで置換*/
     4	#define MAX_LINE 1025                                      /*文字配列LINEの
最大入力数の指定用*/
     5	
     6	int subst(char *str, char c1, char c2)
     7	{
     8	  int i;                                                   /*forループ用*/
     9	  int c = 0;                                               /*置き換えた文字数の
カウント用*/
    10	  for(i = 0; *(str + i) != '\0'; i++)                      /*入力文字列の終端に辿り
着くまでループ*/
    11	    {
    12	      if(c1 == c2) break;                                  /*見た目上文字列に変化が
ないとき*/
    13	      if(*(str + i) == c1)                                 /*(str + i)の文字がc1の
文字と同じとき*/
    14		{
    15		  *(str + i) = c2;                                 /*(str + i)の文字をc2の
文字に置き換える*/
    16		  c++;                                             /*置き換えた文字を数える*/
    17		}
    18	    }
    19	  return c;                                                /*置き換えた文字数を戻り値
とする.*/
    20	}
    21	
    22	int split(char *str, char *ret[], char sep, int max)
    23	{
    24	  int i;                                                   /*forループ用*/
    25	  int c = 0;                                               /*ポインタの配列の指定用*/
    26	
    27	  ret[0] = str;                                            /*ret[0]にstrの先頭アドレス
を代入*/
    28	
    29	  for(i = 0; *(str + i) != '\0'&& c < max; i++)            /*cがmaxより小さいか,入力
文字列の終端に辿り着いていないときループ*/
    30	    {
    31	      if(*(str + i) == sep)                                /*(str + i)がsepのとき*/
    32		{
    33		  *(str + i) = '\0';                               /*(str + i)にNULLを代入*/
    34		  c++;
    35		  ret[c] = str + (i + 1);                          /*ret[c]にNULL文字の"次の"
アドレスを代入*/
    36		}
    37	    }
    38	  return c;                                                /*文字列をいくつに分割したか
を戻り値とする*/
    39	}
    40	
    41	int get_line(char *line)
    42	{
    43	  if(fgets(line, MAX_LINE, stdin) == NULL) return 0;       /*入力文字列が空のとき,
0を戻り値とする.入力文字列は1024文字*/
    44	  if(*line == ESC) return 0;                               /*直接入力のとき,入力文字列
を空にできないため,ESCキーの単独入力により0を戻り値とする*/
    45	  subst(line, '\n', '\0');                                 /*subst関数により,入力の
改行文字を終端文字に置き換える*/
    46	  return 1;                                                /*入力文字列が存在したとき,
1を戻り値とする*/
    47	}
    48	
    49	int main(void)
    50	{
    51	  char line[MAX_LINE] = {0};                               /*入力文字列は最大1024文字*/
    52	  char *ret[10];
    53	  char sep = ',';                                          /*csvファイルからの入力を
想定しているため,カンマ*/
    54	  int max = 10;
    55	  int c, i, a = 1;
    56	  
    57	  while(get_line(line))                                    /*get_line関数を呼び出し,
戻り値が0ならループを終了*/
    58	    {
    59	      printf("line number:%d\n", a);
    60	      printf("input:\"%s\"\n", line);
    61	      c = split(line, ret, sep, max);                      /*split関数を呼び出す*/
    62	      for(i = 0;i <= c; i++)
    63		{
    64		  printf("split[%d]:\"%s\"\n", i, ret[i]);
    65		}
    66	      printf("\n\n");                                      /*見やすさのために改行*/
    67	      a++;
    68	    }
    69	  return 0;
    70	}
\end{Verbatim}

\subsection{プログラミング演習1第3回講義のプログラム}\label{func3}
\verb|parse_line()|,\verb|exec_command|関数を加えたプログラムのソースコード(169行)
\begin{Verbatim}[fontsize=\small, baselinestretch=0.8]
     1	#include <stdio.h>
     2	#include <stdlib.h>                                        /*exit関数用*/
     3	
     4	#define ESC 27                                             /*文字列ESCをESCのASCII
コードで置換*/
     5	#define MAX_LINE 1025                                      /*文字配列LINEの最大入力数
の指定用*/
     6	
     7	/*関数プロトタイプ宣言(煩雑化防止)*/
     8	int subst(char *str, char c1, char c2);
     9	int split(char *str, char *ret[], char sep, int max);
    10	int get_line(char *line);
    11	void parse_line(char *line);
    12	void exec_command(char cmd, char *param);
    13	void cmd_quit();
    14	void cmd_check();
    15	void cmd_print();
    16	void cmd_read();
    17	void cmd_write();
    18	void cmd_find();
    19	void cmd_sort();
    20	void new_profile(char *line);
    21	
    22	int subst(char *str, char c1, char c2)
    23	{
    24	  int i;                                                   /*forループ用*/
    25	  int c = 0;                                               /*置き換えた文字数のカウ
ント用*/
    26	  for(i = 0; *(str + i) != '\0'; i++)                      /*入力文字列の終端に辿り
着くまでループ*/
    27	    {
    28	      if(c1 == c2) break;                                  /*見た目上文字列に変化が
ないとき*/
    29	      if(*(str + i) == c1)                                 /*(str + i)の文字がc1の
文字と同じとき*/
    30		{
    31		  *(str + i) = c2;                                 /*(str + i)の文字をc2の
文字に置き換える*/
    32		  c++;                                             /*置き換えた文字を数える*/
    33		}
    34	    }
    35	  return c;                                                /*置き換えた文字数を戻り値
とする.*/
    36	}
    37	
    38	int split(char *str, char *ret[], char sep, int max)
    39	{
    40	  int i;                                                   /*forループ用*/
    41	  int c = 0;                                               /*ポインタの配列の指定用*/
    42	
    43	  ret[0] = str;                                            /*ret[0]にstrの先頭アドレ
スを代入*/
    44	
    45	  for(i = 0; *(str + i) != '\0'&& c < max; i++)            /*cがmaxより小さいかつ入力
文字列の終端に辿り着いていないときループ*/
    46	    {
    47	      if(*(str + i) == sep)                                /*(str + i)がsepのとき*/
    48		{
    49		  *(str + i) = '\0';                               /*(str + i)にNULLを代入*/
    50		  c++;
    51		  ret[c] = str + (i + 1);                          /*ret[c]にNULL文字の"次の"
アドレスを代入*/
    52		}
    53	    }
    54	  return c;                                                /*文字列をいくつに分割したか
を戻り値とする*/
    55	}
    56	
    57	int get_line(char *line)
    58	{
    59	  if(fgets(line, MAX_LINE, stdin) == NULL) return 0;       /*入力文字列が空のとき,0を
戻り値とする.入力文字列は1024文字*/
    60	  if(*line == ESC) return 0;                               /*ESCを1文字目に入力すること
により0を戻り値とする(デバッグ用)*/
    61	  subst(line, '\n', '\0');                                 /*subst関数により,入力の
改行文字を終端文字に置き換える*/
    62	  return 1;                                                /*入力文字列が存在したとき,
1を戻り値とする*/
    63	}
    64	
    65	void parse_line(char *line)
    66	{
    67	  char cmd;                                                /*%の次の1文字を格納用*/
    68	  char *param;                                             /*コマンドのパラメータとなる
文字列へのポインタ用*/
    69	  char *buffer = "(Null Parameter)";                       /*例外処理用*/
    70	
    71	  if(*line == '%')                                         /*入力文字列の1文字目が%のとき*/
    72	    {
    73	      cmd = *(line + 1);                                   /*cmdに入力文字列の2文字目
の値を代入*/
    74	      if(*(line + 3) != '\0') param = (line + 3);          /*ポインタlineに3を足した
アドレスをポインタparamに代入*/
    75	      else param = buffer;                                 /*入力文字列にパラメータ部が
無いとき,文字列"(Null Parameter)"のアドレスをポインタparamに代入*/
    76	      exec_command(cmd, param);
    77	    }
    78	  else                                                     /*入力がコマンドではないとき*/
    79	    {
    80	      new_profile(line);
    81	    }
    82	}
    83	
    84	void exec_command(char cmd, char *param)
    85	{
    86	  switch (cmd) {
    87	  case 'Q': cmd_quit();   break;
    88	  case 'T': printf("Parameter test:\"%s\"\n", param); break;   /*ポインタparamの参照先
から後ろに向かって,NULLまで文字列を表示する(デバッグ用)*/
    89	  case 'C': cmd_check();  break;
    90	  case 'P': cmd_print();  break;
    91	  case 'R': cmd_read();   break;
    92	  case 'W': cmd_write();  break;
    93	  case 'F': cmd_find();   break;
    94	  case 'S': cmd_sort();   break;
    95	  default: fprintf(stderr, "%%%c command is invoked with arg: \"%s\"\n", cmd, param);
 break;/*エラーメッセージを表示*/
    96	  }
    97	}
    98	
    99	void cmd_quit()
   100	{
   101	  printf("Do you want to quit?(y/n)\n");                   /*確認メッセージ*/
   102	  if(getchar() == 'y')
   103	    {
   104	      printf("quit success.\n");
   105	      exit(0);
   106	    }
   107	  getchar();                                               /*getcharでの入力時に改行文字が
残ってしまうため*/
   108	  printf("quit cancelled.\n");
   109	}
   110	
   111	void cmd_check()
   112	{
   113	  fprintf(stderr, "Check command is invoked.\n");
   114	}
   115	
   116	void cmd_print()
   117	{
   118	  fprintf(stderr, "Print command is invoked.\n");
   119	}
   120	
   121	void cmd_read()
   122	{
   123	  fprintf(stderr, "Read command is invoked.\n");
   124	}
   125	
   126	void cmd_write()
   127	{
   128	  fprintf(stderr, "Write command is invoked.\n");
   129	}
   130	
   131	void cmd_find()
   132	{
   133	  fprintf(stderr, "Find command is invoked.\n");
   134	}
   135	
   136	void cmd_sort()
   137	{
   138	  fprintf(stderr, "Sort command is invoked.\n");
   139	}
   140	
   141	void new_profile(char *line)
   142	{
   143	  char *ret[11];
   144	  char sep = ',';                                          /*csvファイルからの入力を想定して
いるため,カンマ*/
   145	  int max = 10;
   146	  int c, i;
   147	  static int a = 1;                                        /*値をmain関数終了時まで保持する
必要があるため,static int型*/
   148	  
   149	  printf("line number:%d\n", a);
   150	  printf("input:\"%s\"\n", line);
   151	  c = split(line, ret, sep, max);                          /*split関数を呼び出す*/
   152	  for(i = 0;i <= c; i++)
   153	    {
   154	      printf("split[%d]:\"%s\"\n", i, ret[i]);
   155	    }
   156	  printf("\n");                                            /*見やすさのために改行*/
   157	  a++;
   158	}
   159	
   160	int main(void)
   161	{
   162	  char LINE[MAX_LINE] = {0};                               /*入力文字列(1行分)はmain関数で
管理*/
   163	
   164	  while(get_line(LINE))                                    /*文字配列LINEに文字列を入力する
(get_line関数)*/
   165	    {
   166	      parse_line(LINE);                                    /*入力文字列がある場合,構文解析
を行う(parse_line関数)*/
   167	    }
   168	  return 0;
   169	}
\end{Verbatim}

\subsection{プログラミング演習1第4回講義のプログラム}\label{func4}
\verb|profile|,\verb|date|構造体と\verb|new_profile()|関数を加えたプログラムのソースコード(239行)
\begin{Verbatim}[fontsize=\small, baselinestretch=0.8]
     1	#include <stdio.h>
     2	#include <string.h>                                        /*strncpy関数等用*/
     3	#include <stdlib.h>                                        /*exit関数用*/
     4	
     5	#define ESC 27                                             /*文字列ESCをESC
のASCIIコードで置換*/
     6	#define MAX_LINE 1025                                      /*文字配列LINEの
最大入力数の指定用*/
     7	
     8	/*構造体宣言*/
     9	struct date
    10	{
    11	  int y; /*年*/
    12	  int m; /*月*/
    13	  int d; /*日*/
    14	};
    15	
    16	struct profile
    17	{
    18	  int id;               /*ID*/
    19	  char name[70];        /*名前*/
    20	  struct date birthday; /*誕生日(date構造体)*/
    21	  char address[70];     /*住所*/
    22	  char biko[70];        /*備考*/
    23	};
    24	
    25	/*関数プロトタイプ宣言(煩雑化防止)*/
    26	int subst(char *str, char c1, char c2);
    27	int split(char *str, char *ret[], char sep, int max);
    28	int get_line(char *line);
    29	void parse_line(char *line);
    30	void exec_command(char cmd, char *param);
    31	void cmd_quit();
    32	void cmd_check();
    33	void cmd_print();
    34	void cmd_read();
    35	void cmd_write();
    36	void cmd_find();
    37	void cmd_sort();
    38	void new_profile(struct profile *profile_p, char *line);
    39	
    40	/*グローバル変数宣言*/
    41	struct profile profile_data_store[10000];                  /*profile情報を
格納*/
    42	int profile_data_nitems = 0;                               /*profile情報の
保存数を格納*/
    43	
    44	int subst(char *str, char c1, char c2)
    45	{
    46	  int i;                                                   /*forループ用*/
    47	  int c = 0;                                               /*置き換えた文字
数のカウント用*/
    48	  for(i = 0; *(str + i) != '\0'; i++)                      /*入力文字列の
終端に辿り着くまでループ*/
    49	    {
    50	      if(c1 == c2) break;                                  /*見た目上文字列
に変化がないとき*/
    51	      if(*(str + i) == c1)                                 /*(str + i)の
文字がc1の文字と同じとき*/
    52		{
    53		  *(str + i) = c2;                                 /*(str + i)の
文字をc2の文字に置き換える*/
    54		  c++;                                             /*置き換えた文字
を数える*/
    55		}
    56	    }
    57	  return c;                                                /*置き換えた文字
数を戻り値とする.*/
    58	}
    59	
    60	int split(char *str, char *ret[], char sep, int max)
    61	{
    62	  int i;                                                   /*forループ用*/
    63	  int c = 0;                                               /*ポインタの配列
の指定用*/
    64	
    65	  ret[0] = str;                                            /*ret[0]にstr
の先頭アドレスを代入*/
    66	
    67	  for(i = 0; *(str + i) != '\0'&& c < max; i++)            /*cがmaxより小
さいかつ入力文字列の終端に辿り着いていないときループ*/
    68	    {
    69	      if(*(str + i) == sep)                                /*(str + i)が
sepのとき*/
    70		{
    71		  *(str + i) = '\0';                               /*(str + i)に
NULLを代入*/
    72		  c++;
    73		  ret[c] = str + (i + 1);                          /*ret[c]にNULL
文字の"次の"アドレスを代入*/
    74		}
    75	    }
    76	  return c;                                                /*文字列をいく
つに分割したかを戻り値とする*/
    77	}
    78	
    79	int get_line(char *line)
    80	{
    81	  if(fgets(line, MAX_LINE, stdin) == NULL) return 0;       /*入力文字列が
空のとき,0を戻り値とする.入力文字列は1024文字*/
    82	  if(*line == ESC) return 0;                               /*直接入力の
とき,入力文字列を空にできないため,ESCキーの単独入力により0を戻り値とする(デバッグ用)*/
    83	  subst(line, '\n', '\0');                                 /*subst関数に
より,入力の改行文字を終端文字に置き換える*/
    84	  return 1;                                                /*入力文字列が
存在したとき,1を戻り値とする*/
    85	}
    86	
    87	void parse_line(char *line)
    88	{
    89	  char cmd;                                                /*%の次の1文字
を格納用*/
    90	  char *param;                                             /*コマンドの
パラメータとなる文字列へのポインタ用*/
    91	  char *buffer = "(Null Parameter)";                       /*例外処理用*/
    92	
    93	  if(*line == '%')                                         /*入力文字列
の1文字目が%のとき*/
    94	    {
    95	      cmd = *(line + 1);                                   /*cmdに入力
文字列の2文字目の値を代入*/
    96	      if(*(line + 3) != '\0')                              /*パラメータ部
があるとき*/
    97		{
    98		  if(*(line + 2) != ' ')                           /*3文字目が空白
でないとき*/
    99		    {
   100		      param = line + 2;
   101		      printf("引数を要するコマンド入力の場合,3文字目は空白である必要が
あります.\n処理を中止しました.\n\n");
   102		      return;
   103		    }
   104		  else
   105		  param = line + 3;                                /*ポインタline
に3を足したアドレスをポインタparamに代入*/
   106		}
   107	      else param = buffer;                                 /*入力文字列に
パラメータ部が無いとき,文字列"(Null Parameter)"のアドレスをポインタparamに代入*/
   108	      exec_command(cmd, param);
   109	    }
   110	  else                                                     /*入力がコマンド
ではないとき*/
   111	    {
   112	      new_profile(&profile_data_store[profile_data_nitems++] ,line);
   113	    }
   114	}
   115	
   116	void exec_command(char cmd, char *param)
   117	{
   118	  switch (cmd) {
   119	  case 'Q': cmd_quit();   break;
   120	  case 'T': printf("Parameter test:\"%s\"\n", param); break;   /*ポインタparam
の参照先から後ろに向かって,NULLまで文字列を表示する(デバッグ用)*/
   121	  case 'C': cmd_check();  break;
   122	  case 'P': cmd_print();  break;
   123	  case 'R': cmd_read();   break;
   124	  case 'W': cmd_write();  break;
   125	  case 'F': cmd_find();   break;
   126	  case 'S': cmd_sort();   break;
   127	  default: fprintf(stderr, "%%%c command is invoked with arg: \"%s\"\n", cmd, param); 
break;/*エラーメッセージを表示*/
   128	  }
   129	}
   130	
   131	void cmd_quit()
   132	{
   133	  char c;
   134	
   135	  while(1)
   136	    {
   137	      printf("Do you want to quit?(y/n)\n");               /*確認メッセージ*/
   138	      c = getchar();
   139	      getchar();                                           /*getcharでの入力
時に改行文字が残ってしまうため*/
   140	      if(c == 'y')
   141		{
   142		  printf("quit success.\n\n");
   143		  exit(0);
   144		}
   145	      else if(c == 'n')
   146		{
   147		  printf("quit cancelled.\n\n");
   148		  break;
   149		}
   150	    }
   151	}
   152	
   153	void cmd_check()
   154	{
   155	  fprintf(stderr, "Check command is invoked.\n");
   156	}
   157	
   158	void cmd_print()
   159	{
   160	  fprintf(stderr, "Print command is invoked.\n");
   161	}
   162	
   163	void cmd_read()
   164	{
   165	  fprintf(stderr, "Read command is invoked.\n");
   166	}
   167	
   168	void cmd_write()
   169	{
   170	  fprintf(stderr, "Write command is invoked.\n");
   171	}
   172	
   173	void cmd_find()
   174	{
   175	  fprintf(stderr, "Find command is invoked.\n");
   176	}
   177	
   178	void cmd_sort()
   179	{
   180	  fprintf(stderr, "Sort command is invoked.\n");
   181	}
   182	
   183	void new_profile(struct profile *profile_p, char *line)
   184	{
   185	  char *ret[10];
   186	  char *ret2[4];                                           /*誕生日の情報
を分割し,その先頭アドレスを保存*/
   187	  char sep = ',';                                          /*csvファイル
からの入力を想定しているため,カンマ*/
   188	  char sep2 = '-';                                         /*誕生日の入力
文字列にあるハイフンで区切るため,ハイフン*/
   189	  int max = 10;
   190	  int max2 = 4;
   191	  int c, birth_c;
   192	  static int a = 1;                                        /*値をmain関数
終了時まで保持する必要があるため,static int型*/
   193	  
   194	  printf("line number:%d\n", a);                           /*行番号表示(デ
バッグ用)*/
   195	  //printf("input:\"%s\"\n", line);                        /*入力文字列をそ
のまま表示(デバッグ用)*/
   196	  c = split(line, ret, sep, max);                          /*ID,名前などの
情報を分割する*/
   197	  if(c <= 2)                                               /*備考以外の入力
がない場合*/
   198	    {
   199	      printf("情報はID,名前,誕生日,住所,備考の順で入力される必要があり,備考以外
は必須項目です.\n処理を中止しました.\n\n");
   200	      profile_data_nitems--;                               /*処理中止により,
構造体に情報を書き込まないため*/
   201	      return;
   202	    }
   203	  birth_c = split(ret[2], ret2, sep2, max2);               /*誕生日の年,月,
日を分割する*/
   204	  if(birth_c != 2)                                         /*誕生日の年,月,
日を正常に分割できない場合*/
   205	    {
   206	      printf("誕生日は\"年-月-日\"の形で入力される必要があります.\n処理を中止しまし
た.\n\n"); /*年,月,日に分割できない場合,処理を停止*/
   207	      profile_data_nitems--;                               /*処理中止により,
構造体に情報を書き込まないため*/
   208	      return;
   209	    }
   210	
   211	  /*構造体への情報の書き込み処理*/
   212	  profile_p->id = atoi(ret[0]);                            /*IDの書き込み*/
   213	  strncpy(profile_p->name, ret[1], 70);                    /*名前の書き込み*/
   214	  (profile_p->birthday).y = atoi(ret2[0]);                 /*誕生年の書き込み*/
   215	  (profile_p->birthday).m = atoi(ret2[1]);                 /*誕生月の書き込み*/
   216	  (profile_p->birthday).d = atoi(ret2[2]);                 /*誕生日の書き込み*/
   217	  strncpy(profile_p->address, ret[3], 70);                 /*住所の書き込み*/
   218	  if(ret[4] != NULL)
   219	    strncpy(profile_p->biko, ret[4], 70);                  /*備考があるときのみ,
備考の書き込み*/
   220	
   221	  printf("id:\"%d\"\n", profile_p->id);
   222	  printf("name:\"%s\"\n", profile_p->name);
   223	  printf("birthday:\"%d-%d-%d\"\n", (profile_p->birthday).y, 
(profile_p->birthday).m, (profile_p->birthday).d);
   224	  printf("address:\"%s\"\n",profile_p->address);
   225	  printf("biko:\"%s\"\n\n",profile_p->biko);
   226	
   227	  a++;
   228	}
   229	
   230	int main(void)
   231	{
   232	  char LINE[MAX_LINE] = {0};                               /*入力文字列(1行分)は
main関数で管理*/
   233	
   234	  while(get_line(LINE))                                    /*文字配列LINEに文字列
を入力する(get_line関数)*/
   235	    {
   236	      parse_line(LINE);                                    /*入力文字列がある場合,
構文解析を行う(parse_line関数)*/
   237	    }
   238	  return 0;
   239	}

\end{Verbatim}

\subsection{プログラミング演習1第5回講義のプログラム}\label{func5}

本プログラムは,\verb|UNIX|用\verb|db-sample|に合わせた仕様となっている.また,プログラムの一部機能を\verb|//|で無効化している.

\verb|C|,\verb|P|コマンドを加えたプログラムのソースコード(248行)

\begin{Verbatim}[fontsize=\small, baselinestretch=0.8]
     1	#include <stdio.h>
     2	#include <string.h>                                        /*strncpy関数等用*/
     3	#include <stdlib.h>                                        /*exit関数用*/
     4	
     5	#define ESC 27                                             /*文字列ESCをESC
のASCIIコードで置換*/
     6	#define MAX_LINE 1025                                      /*文字配列LINEの
最大入力数の指定用*/
     7	
     8	/*構造体宣言*/
     9	struct date
    10	{
    11	  int y; /*年*/
    12	  int m; /*月*/
    13	  int d; /*日*/
    14	};
    15	
    16	struct profile
    17	{
    18	  int id;               /*ID*/
    19	  char name[70];        /*名前*/
    20	  struct date birthday; /*誕生日(date構造体)*/
    21	  char address[70];     /*住所*/
    22	  char *biko;           /*備考*/
    23	};
    24	
    25	/*関数プロトタイプ宣言(煩雑化防止)*/
    26	int subst(char *str, char c1, char c2);
    27	int split(char *str, char *ret[], char sep, int max);
    28	int get_line(char *line);
    29	void parse_line(char *line);
    30	void exec_command(char cmd, char *param);
    31	void cmd_quit(void);
    32	void cmd_check(void);
    33	void cmd_print(char *param);
    34	void cmd_read();
    35	void cmd_write();
    36	void cmd_find();
    37	void cmd_sort();
    38	void new_profile(struct profile *profile_p, char *line);
    39	
    40	/*グローバル変数宣言*/
    41	struct profile profile_data_store[10000];                  /*profile情報を
格納*/
    42	int profile_data_nitems = 0;                               /*profile情報の
保存数を格納*/
    43	
    44	int subst(char *str, char c1, char c2)
    45	{
    46	  int i;                                                   /*forループ用*/
    47	  int c = 0;                                               /*置き換えた文字
数のカウント用*/
    48	  for(i = 0; *(str + i) != '\0'; i++)                      /*入力文字列の
終端に辿り着くまでループ*/
    49	    {
    50	      if(c1 == c2) break;                                  /*見た目上文字列
に変化がないとき*/
    51	      if(*(str + i) == c1)                                 /*(str + i)の
文字がc1の文字と同じとき*/
    52		{
    53		  *(str + i) = c2;                                 /*(str + i)の
文字をc2の文字に置き換える*/
    54		  c++;                                             /*置き換えた文字
を数える*/
    55		}
    56	    }
    57	  return c;                                                /*置き換えた文字
数を戻り値とする.*/
    58	}
    59	
    60	int split(char *str, char *ret[], char sep, int max)
    61	{
    62	  int i;                                                   /*forループ用*/
    63	  int c = 0;                                               /*ポインタの配列
の指定用*/
    64	
    65	  ret[0] = str;                                            /*ret[0]にstr
の先頭アドレスを代入*/
    66	
    67	  for(i = 0; *(str + i) != '\0'&& c < max; i++)            /*cがmaxより小
さいかつ入力文字列の終端に辿り着いていないときループ*/
    68	    {
    69	      if(*(str + i) == sep)                                /*(str + i)が
sepのとき*/
    70		{
    71		  *(str + i) = '\0';                               /*(str + i)に
NULLを代入*/
    72		  c++;
    73		  ret[c] = str + (i + 1);                          /*ret[c]にNULL
文字の"次の"アドレスを代入*/
    74		}
    75	    }
    76	  return c;                                                /*文字列をいく
つに分割したかを戻り値とする*/
    77	}
    78	
    79	int get_line(char *line)
    80	{
    81	  if(fgets(line, MAX_LINE, stdin) == NULL) return 0;       /*入力文字列が
空のとき,0を戻り値とする.入力文字列は1024文字*/
    82	  if(*line == ESC) cmd_quit();                             /*デバッグ用*/
    83	  subst(line, '\n', '\0');                                 /*subst関数に
より,入力の改行文字を終端文字に置き換える*/
    84	  return 1;                                                /*入力文字列が
存在したとき,1を戻り値とする*/
    85	}
    86	
    87	void parse_line(char *line)
    88	{
    89	  char cmd;                                                /*%の次の1文字
を格納用*/
    90	  char *param;                                             /*コマンドの
パラメータとなる文字列へのポインタ用*/
    91	
    92	  if(*line == '%')                                         /*入力文字列
の1文字目が%のとき*/
    93	    {
    94	      cmd = *(line + 1);                                   /*cmdに入力
文字列の2文字目の値を代入*/
    95	      param = line + 3;                                    /*paramにパラメータ
部を代入*/
    96	      exec_command(cmd, param);
    97	    }
    98	  else                                                     /*入力がコマンド
ではないとき*/
    99	    {
   100	      new_profile(&profile_data_store[profile_data_nitems++] ,line);
   101	    }
   102	}
   103	
   104	void exec_command(char cmd, char *param)
   105	{
   106	  switch (cmd) {
   107	  case 'Q': cmd_quit();   break;
   108	    //case 'T': printf("Parameter test:\"%s\"\n", param); break;   /*ポインタparam
の参照先から後ろに向かって,NULLまで文字列を表示する(デバッグ用)*/
   109	  case 'C': cmd_check();  break;
   110	  case 'P': cmd_print(param);  break;
   111	  case 'R': cmd_read();   break;
   112	  case 'W': cmd_write();  break;
   113	  case 'F': cmd_find();   break;
   114	  case 'S': cmd_sort();   break;
   115	  default: fprintf(stderr, "Invalid command %c: ignored.\n", cmd); break;/*エラー
メッセージを表示*/
   116	  }
   117	}
   118	
   119	void cmd_quit()
   120	{
   121	  //char c;
   122	
   123	  //while(1)
   124	  //{
   125	  //printf("Do you want to quit?(y/n)\n");               /*確認メッセージ*/
   126	  //c = getchar();
   127	  //getchar();                                           /*getcharでの入力
時に改行文字が残ってしまうため*/
   128	  //if(c == 'y')
   129	  //{
   130	  //printf("quit success.\n\n");
   131		  exit(0);
   132		  //}
   133		  //else if(c == 'n')
   134		  //{
   135		  //printf("quit cancelled.\n\n");
   136		  //break;
   137		  //}
   138		  //}
   139	}
   140	
   141	void cmd_check(void)
   142	{
   143	  printf("%d profile(s)\n", profile_data_nitems);
   144	}
   145	
   146	void cmd_print(char *param)
   147	{
   148	  int a = 0;
   149	  int i = 0;                                                 /*forループ用*/
   150	
   151	  a = atoi(param);                                           /*文字列をint型の値に変換*/
   152	
   153	  /*aの絶対値がprofile_data_nitemsより大きいときかa=0のとき*/
   154	  if(abs(a) >= profile_data_nitems|| a == 0) a = profile_data_nitems;
   155	
   156	  if(a > 0)                                                  /*引数が正の整数のとき及び例外*/
   157	    {
   158	      for(i = 0; i < a; i++)
   159	        {
   160		  printf("Id    : %d\n", profile_data_store[i].id);
   161		  printf("Name  : %s\n", profile_data_store[i].name);
   162		  printf("Birth : %04d-%02d-%02d\n", profile_data_store[i].birthday.y, profile_
data_store[i].birthday.m, profile_data_store[i].birthday.d);
   163		  printf("Addr. : %s\n",profile_data_store[i].address);
   164		  printf("Comm. : %s\n\n",profile_data_store[i].biko);
   165		}
   166	    }
   167	  else if(a < 0)                                             /*引数が負の整数のとき*/
   168	    {
   169	      for(i = profile_data_nitems + a; i < profile_data_nitems; i++)
   170		{
   171		  printf("Id    : %d\n", profile_data_store[i].id);
   172		  printf("Name  : %s\n", profile_data_store[i].name);
   173		  printf("Birth : %04d-%02d-%02d\n", profile_data_store[i].birthday.y, profile_
data_store[i].birthday.m, profile_data_store[i].birthday.d);
   174		  printf("Addr. : %s\n",profile_data_store[i].address);
   175		  printf("Comm. : %s\n\n",profile_data_store[i].biko);
   176		}
   177	    }
   178	}
   179	
   180	void cmd_read()
   181	{
   182	  fprintf(stderr, "Read command is invoked.\n");
   183	}
   184	
   185	void cmd_write()
   186	{
   187	  fprintf(stderr, "Write command is invoked.\n");
   188	}
   189	void cmd_find()
   190	{
   191	  fprintf(stderr, "Find command is invoked.\n");
   192	}
   193	
   194	void cmd_sort()
   195	{
   196	  fprintf(stderr, "Sort command is invoked.\n");
   197	}
   198	
   199	void new_profile(struct profile *profile_p, char *line)
   200	{
   201	  char *ret[10];
   202	  char *ret2[4];                                           /*誕生日の情報
を分割し,その先頭アドレスを保存*/
   203	  char sep = ',';                                          /*csvファイル
からの入力を想定しているため,カンマ*/
   204	  char sep2 = '-';                                         /*誕生日の入力
文字列にあるハイフンで区切るため,ハイフン*/
   205	  int max = 10;
   206	  int max2 = 4;
   207	  int c, birth_c;
   208	  int MAX_BIKO = 0;                                        /*備考の文字数
カウント用*/
   209	  
   210	  c = split(line, ret, sep, max);                          /*ID,名前などの
情報を分割する*/
   211	  if(c != 4)                                               /*入力形式が合わ
ない場合*/
   212	    {
   213	      fprintf(stderr, "情報はID,名前,誕生日,住所,備考の順で入力される必要があり
ます.\n処理を中止しました.\n\n");
   214	      profile_data_nitems--;                               /*処理中止により,
構造体に情報を書き込まないため*/
   215	      return;
   216	    }
   217	  birth_c = split(ret[2], ret2, sep2, max2);               /*誕生日の年,月,
日を分割する*/
   218	  if(birth_c != 2)                                         /*誕生日の年,月,
日を正常に分割できない場合*/
   219	      {
   220		fprintf(stderr, "誕生日は\"年-月-日\"の形で入力される必要があります.\n処理を中止
しました.\n\n"); /*年,月,日に分割できない場合,処理を停止*/
   221		profile_data_nitems--;                             /*処理中止により,
構造体に情報を書き込まないため*/
   222		return;
   223	      }
   224	
   225	  /*構造体への情報の書き込み処理*/
   226	  profile_p->id = atoi(ret[0]);                            /*IDの書き込み*/
   227	  strncpy(profile_p->name, ret[1], 69);                    /*名前の書き込み*/
   228	  (profile_p->birthday).y = atoi(ret2[0]);                 /*誕生年の書き込み*/
   229	  (profile_p->birthday).m = atoi(ret2[1]);                 /*誕生月の書き込み*/
   230	  (profile_p->birthday).d = atoi(ret2[2]);                 /*誕生日の書き込み*/
   231	  strncpy(profile_p->address, ret[3], 69);                 /*住所の書き込み*/
   232	
   233	  MAX_BIKO = strlen(ret[4]) + 1;                           /*備考情報の文字数
のカウント*/
   234	
   235	  profile_p->biko = (char *)malloc(sizeof(char) * MAX_BIKO); /*文字数分だけ
メモリ確保*/
   236	  strncpy(profile_p->biko, ret[4], MAX_BIKO);              /*備考の書き込み*/
   237	}
   238	
   239	int main(void)
   240	{
   241	  char LINE[MAX_LINE] = {0};                               /*入力文字列(1行分)は
main関数で管理*/
   242	
   243	  while(get_line(LINE))                                    /*文字配列LINEに文字列
を入力する*/
   244	    {
   245	      parse_line(LINE);                                    /*入力文字列がある場合,
構文解析を行う*/
   246	    }
   247	  return 0;
   248	}
\end{Verbatim}

\subsection{プログラミング演習1最終プログラム}\label{func6}

プログラミング演習1の最後の基本関数のテストで\verb|split()|関数がNGとなったため,ソースコードに修正を加えた.

\verb|split()|関数に修正を加えたプログラムのソースコード(247行)

\begin{Verbatim}[fontsize=\small, baselinestretch=0.8]
     1	#include <stdio.h>
     2	#include <string.h>                                        /*strncpy関数等用*/
     3	#include <stdlib.h>                                        /*exit関数用*/
     4	
     5	#define ESC 27                                             /*文字列ESCをESC
のASCIIコードで置換*/
     6	#define MAX_LINE 1025                                      /*文字配列LINEの
最大入力数の指定用*/
     7	
     8	/*構造体宣言*/
     9	struct date
    10	{
    11	  int y; /*年*/
    12	  int m; /*月*/
    13	  int d; /*日*/
    14	};
    15	
    16	struct profile
    17	{
    18	  int id;               /*ID*/
    19	  char name[70];        /*名前*/
    20	  struct date birthday; /*誕生日(date構造体)*/
    21	  char address[70];     /*住所*/
    22	  char *biko;           /*備考*/
    23	};
    24	
    25	/*関数プロトタイプ宣言(煩雑化防止)*/
    26	int subst(char *str, char c1, char c2);
    27	int split(char *str, char *ret[], char sep, int max);
    28	int get_line(char *line);
    29	void parse_line(char *line);
    30	void exec_command(char cmd, char *param);
    31	void cmd_quit(void);
    32	void cmd_check(void);
    33	void cmd_print(char *param);
    34	void cmd_read();
    35	void cmd_write();
    36	void cmd_find();
    37	void cmd_sort();
    38	void new_profile(struct profile *profile_p, char *line);
    39	
    40	/*グローバル変数宣言*/
    41	struct profile profile_data_store[10000];                  /*profile情報を
格納*/
    42	int profile_data_nitems = 0;                               /*profile情報の
保存数を格納*/
    43	
    44	int subst(char *str, char c1, char c2)
    45	{
    46	  int i;                                                   /*forループ用*/
    47	  int c = 0;                                               /*置き換えた文字
数のカウント用*/
    48	  for(i = 0; *(str + i) != '\0'; i++)                      /*入力文字列の
終端に辿り着くまでループ*/
    49	    {
    50	      if(c1 == c2) break;                                  /*見た目上文字列
に変化がないとき*/
    51	      if(*(str + i) == c1)                                 /*(str + i)の
文字がc1の文字と同じとき*/
    52		{
    53		  *(str + i) = c2;                                 /*(str + i)の
文字をc2の文字に置き換える*/
    54		  c++;                                             /*置き換えた文字
を数える*/
    55		}
    56	    }
    57	  return c;                                                /*置き換えた文字
数を戻り値とする.*/
    58	}
    59	
    60	int split(char *str, char *ret[], char sep, int max)
    61	{
    62	  int i;                                                   /*forループ用*/
    63	  int c = 0;                                               /*ポインタの配列
の指定用*/
    64	
    65	  ret[c++] = str;                                            /*ret[0]にstr
の先頭アドレスを代入*/
    66	
    67	  for(i = 0; *(str + i) != '\0'&& c < max; i++)            /*cがmaxより小
さいかつ入力文字列の終端に辿り着いていないときループ*/
    68	    {
    69	      if(*(str + i) == sep)                                /*(str + i)が
sepのとき*/
    70		{
    71		  *(str + i) = '\0';                               /*(str + i)に
NULLを代入*/
    72		  ret[c++] = str + (i + 1);                          /*ret[c]にNULL
文字の"次の"アドレスを代入*/
    73		}
    74	    }
    75	  return c;                                                /*文字列をいく
つに分割したかを戻り値とする*/
    76	}
    77	
    78	int get_line(char *line)
    79	{
    80	  if(fgets(line, MAX_LINE, stdin) == NULL) return 0;       /*入力文字列が
空のとき,0を戻り値とする.入力文字列は1024文字*/
    81	  if(*line == ESC) cmd_quit();
    82	  subst(line, '\n', '\0');                                 /*subst関数に
より,入力の改行文字を終端文字に置き換える*/
    83	  return 1;                                                /*入力文字列が
存在したとき,1を戻り値とする*/
    84	}
    85	
    86	void parse_line(char *line)
    87	{
    88	  char cmd;                                                /*%の次の1文字
を格納用*/
    89	  char *param;                                             /*コマンドの
パラメータとなる文字列へのポインタ用*/
    90	
    91	  if(*line == '%')                                         /*入力文字列
の1文字目が%のとき*/
    92	    {
    93	      cmd = *(line + 1);                                   /*cmdに入力
文字列の2文字目の値を代入*/
    94	      param = line + 3;                                    /*paramにパラメータ
部を代入*/
    95	      exec_command(cmd, param);
    96	    }
    97	  else                                                     /*入力がコマンド
ではないとき*/
    98	    {
    99	      new_profile(&profile_data_store[profile_data_nitems++] ,line);
   100	    }
   101	}
   102	
   103	void exec_command(char cmd, char *param)
   104	{
   105	  switch (cmd) {
   106	  case 'Q': cmd_quit();   break;
   107	    //case 'T': printf("Parameter test:\"%s\"\n", param); break;   /*ポインタparam
の参照先から後ろに向かって,NULLまで文字列を表示する(デバッグ用)*/
   108	  case 'C': cmd_check();  break;
   109	  case 'P': cmd_print(param);  break;
   110	  case 'R': cmd_read();   break;
   111	  case 'W': cmd_write();  break;
   112	  case 'F': cmd_find();   break;
   113	  case 'S': cmd_sort();   break;
   114	  default: fprintf(stderr, "Invalid command %c: ignored.\n", cmd); break;/*エラー
メッセージを表示*/
   115	  }
   116	}
   117	
   118	void cmd_quit()
   119	{
   120	  //char c;
   121	
   122	  //while(1)
   123	  //{
   124	  //printf("Do you want to quit?(y/n)\n");               /*確認メッセージ*/
   125	  //c = getchar();
   126	  //getchar();                                           /*getcharでの入力
時に改行文字が残ってしまうため*/
   127	  //if(c == 'y')
   128	  //{
   129	  //printf("quit success.\n\n");
   130		  exit(0);
   131		  //}
   132		  //else if(c == 'n')
   133		  //{
   134		  //printf("quit cancelled.\n\n");
   135		  //break;
   136		  //}
   137		  //}
   138	}
   139	
   140	void cmd_check(void)
   141	{
   142	  printf("%d profile(s)\n", profile_data_nitems);
   143	}
   144	
   145	void cmd_print(char *param)
   146	{
   147	  int a = 0;
   148	  int i = 0;                                                 /*forループ用*/
   149	
   150	  a = atoi(param);                                           /*文字列をint型の値に変換*/
   151	
   152	  /*aの絶対値がprofile_data_nitemsより大きいときかa=0のとき*/
   153	  if(abs(a) >= profile_data_nitems|| a == 0) a = profile_data_nitems;
   154	
   155	  if(a > 0)                                                  /*引数が正の整数のとき及び例外*/
   156	    {
   157	      for(i = 0; i < a; i++)
   158	        {
   159		  printf("Id    : %d\n", profile_data_store[i].id);
   160		  printf("Name  : %s\n", profile_data_store[i].name);
   161		  printf("Birth : %04d-%02d-%02d\n", profile_data_store[i].birthday.y, profile_
data_store[i].birthday.m, profile_data_store[i].birthday.d);
   162		  printf("Addr. : %s\n",profile_data_store[i].address);
   163		  printf("Comm. : %s\n\n",profile_data_store[i].biko);
   164		}
   165	    }
   166	  else if(a < 0)                                             /*引数が負の整数のとき*/
   167	    {
   168	      for(i = profile_data_nitems + a; i < profile_data_nitems; i++)
   169		{
   170		  printf("Id    : %d\n", profile_data_store[i].id);
   171		  printf("Name  : %s\n", profile_data_store[i].name);
   172		  printf("Birth : %04d-%02d-%02d\n", profile_data_store[i].birthday.y, profile_
data_store[i].birthday.m, profile_data_store[i].birthday.d);
   173		  printf("Addr. : %s\n",profile_data_store[i].address);
   174		  printf("Comm. : %s\n\n",profile_data_store[i].biko);
   175		}
   176	    }
   177	}
   178	
   179	void cmd_read()
   180	{
   181	  fprintf(stderr, "Read command is invoked.\n");
   182	}
   183	
   184	void cmd_write()
   185	{
   186	  fprintf(stderr, "Write command is invoked.\n");
   187	}
   188	void cmd_find()
   189	{
   190	  fprintf(stderr, "Find command is invoked.\n");
   191	}
   192	
   193	void cmd_sort()
   194	{
   195	  fprintf(stderr, "Sort command is invoked.\n");
   196	}
   197	
   198	void new_profile(struct profile *profile_p, char *line)
   199	{
   200	  char *ret[10];
   201	  char *ret2[4];                                           /*誕生日の情報
を分割し,その先頭アドレスを保存*/
   202	  char sep = ',';                                          /*csvファイル
からの入力を想定しているため,カンマ*/
   203	  char sep2 = '-';                                         /*誕生日の入力
文字列にあるハイフンで区切るため,ハイフン*/
   204	  int max = 10;
   205	  int max2 = 4;
   206	  int c, birth_c;
   207	  int MAX_BIKO = 0;                                        /*備考の文字数
カウント用*/
   208	  
   209	  c = split(line, ret, sep, max);                          /*ID,名前などの
情報を分割する*/
   210	  if(c != 5)                                               /*入力形式が合わ
ない場合*/
   211	    {
   212	      fprintf(stderr, "情報はID,名前,誕生日,住所,備考の順で入力される必要があり
ます.\n処理を中止しました.\n\n");
   213	      profile_data_nitems--;                               /*処理中止により,
構造体に情報を書き込まないため*/
   214	      return;
   215	    }
   216	  birth_c = split(ret[2], ret2, sep2, max2);               /*誕生日の年,月,
日を分割する*/
   217	  if(birth_c != 3)                                         /*誕生日の年,月,
日を正常に分割できない場合*/
   218	      {
   219		fprintf(stderr, "誕生日は\"年-月-日\"の形で入力される必要があります.\n処理を中止
しました.\n\n"); /*年,月,日に分割できない場合,処理を停止*/
   220		profile_data_nitems--;                             /*処理中止により,
構造体に情報を書き込まないため*/
   221		return;
   222	      }
   223	
   224	  /*構造体への情報の書き込み処理*/
   225	  profile_p->id = atoi(ret[0]);                            /*IDの書き込み*/
   226	  strncpy(profile_p->name, ret[1], 69);                    /*名前の書き込み*/
   227	  (profile_p->birthday).y = atoi(ret2[0]);                 /*誕生年の書き込み*/
   228	  (profile_p->birthday).m = atoi(ret2[1]);                 /*誕生月の書き込み*/
   229	  (profile_p->birthday).d = atoi(ret2[2]);                 /*誕生日の書き込み*/
   230	  strncpy(profile_p->address, ret[3], 69);                 /*住所の書き込み*/
   231	
   232	  MAX_BIKO = strlen(ret[4]) + 1;                           /*備考情報の文字数
のカウント*/
   233	
   234	  profile_p->biko = (char *)malloc(sizeof(char) * MAX_BIKO); /*文字数分だけ
メモリ確保*/
   235	  strncpy(profile_p->biko, ret[4], MAX_BIKO);              /*備考の書き込み*/
   236	}
   237	
   238	int main(void)
   239	{
   240	  char LINE[MAX_LINE] = {0};                               /*入力文字列(1行分)は
main関数で管理*/
   241	
   242	  while(get_line(LINE))                                    /*文字配列LINEに文字列
を入力する*/
   243	    {
   244	      parse_line(LINE);                                    /*入力文字列がある場合,
構文解析を行う*/
   245	    }
   246	  return 0;
   247	}
\end{Verbatim}

\subsection{プログラミング演習2最終プログラム}\label{funclast}
プログラムのソースコード(627行)
\begin{Verbatim}[fontsize=\small, baselinestretch=0.8]
     1	#include <stdio.h>
     2	#include <string.h>                            /*strncpy関数等用*/
     3	#include <stdlib.h>                            /*exit関数用*/
     4	
     5	#define ESC 27                                 /*文字列ESCをESCのASCIIコードで置換
*/
     6	#define TAB 9                                  /*文字列TABをTABのASCIIコードで置換
*/
     7	#define MAX_LINE 1025                          /*文字配列LINEの最大入力数の指定用*
/
     8	
     9	/*構造体宣言*/
    10	struct date
    11	{
    12	  int y; /*年*/
    13	  int m; /*月*/
    14	  int d; /*日*/
    15	};
    16	
    17	struct profile
    18	{
    19	  int id;               /*ID*/
    20	  char name[70];        /*名前*/
    21	  struct date birthday; /*誕生日(date構造体)*/
    22	  char address[70];     /*住所*/
    23	  char *biko;           /*備考*/
    24	};
    25	
    26	/*関数プロトタイプ宣言(煩雑化防止)*/
    27	int subst(char *str, char c1, char c2);
    28	int split(char *str, char *ret[], char sep, int max);
    29	int get_line(FILE *F, char *line);
    30	void parse_line(char *line);
    31	void exec_command(char *cmd, char *param);
    32	void cmd_quit(char *param);
    33	void cmd_check(void);
    34	void cmd_print(char *param);
    35	void cmd_read(char *param);
    36	void cmd_write(char *param, char sep);
    37	void cmd_find(char *param);
    38	void cmd_sort(char *param);
    39	void data_move(struct profile *sp1, struct profile *sp2);
    40	void new_profile(struct profile *profile_p, char *line);
    41	int int_value_check(char *str);
    42	char input_format_check(char *str);
    43	int day_format_check(int y, int m, int d);
    44	
    45	/*グローバル変数宣言*/
    46	struct profile profile_data_store[10000];      /*profile情報を格納*/
    47	int profile_data_nitems = 0;                   /*profile情報の保存数を格納*/
    48	
    49	int subst(char *str, char c1, char c2)
    50	{
    51	  int i;                                       /*forループ用*/
    52	  int c = 0;                                   /*置き換えた文字数のカウント用*/
    53	  for(i = 0; *(str + i) != '\0'; i++)          /*入力文字列の終端に辿り着くまでル
ープ*/
    54	    {
    55	      if(c1 == c2) break;                      /*見た目上文字列に変化がないとき*/
    56	      if(*(str + i) == c1)                     /*(str + i)の文字がc1の文字と同じと
き*/
    57		{
    58		  *(str + i) = c2;                     /*(str + i)の文字をc2の文字に置き換
える*/
    59		  c++;                                 /*置き換えた文字を数える*/
    60		}
    61	    }
    62	  return c;                                    /*置き換えた文字数を戻り値とする.*
/
    63	}
    64	
    65	int split(char *str, char *ret[], char sep, int max)
    66	{
    67	  int i;                                       /*forループ用*/
    68	  int c = 0;                                   /*ポインタの配列の指定用*/
    69	
    70	  ret[c++] = str;                              /*ret[0]にstrの先頭アドレスを代入*/
    71	
    72	  for(i = 0; *(str + i) != '\0'&& c < max; i++)/*cがmaxより小さいかつ入力文字列の
終端に辿り着いていないときループ*/
    73	    {
    74	      if(*(str + i) == sep)                    /*(str + i)がsepのとき*/
    75		{
    76		  *(str + i) = '\0';                   /*(str + i)にNULLを代入*/
    77		  ret[c++] = str + (i + 1);            /*ret[c]にNULL文字の"次の"アドレス
を代入*/
    78		}
    79	    }
    80	  return c;                                    /*文字列をいくつに分割したかを戻り
値とする*/
    81	}
    82	
    83	int get_line(FILE *F, char *line)
    84	{
    85	
    86	  if(fgets(line, MAX_LINE, F) == NULL) return 0;/*入力文字列が空のとき,0を戻り値
とする.入力文字列は1024文字*/
    87	  if(*line == ESC) cmd_quit("r");
    88	
    89	  subst(line, '\n', '\0');                     /*subst関数により,入力の改行文字を
終端文字に置き換える*/
    90	  return 1;                                    /*入力文字列が存在したとき,1を戻り
値とする*/
    91	}
    92	
    93	void parse_line(char *line)
    94	{
    95	  static int i = 1;                            /*登録中止数カウント用*/
    96	  char *ret[2] = {NULL, NULL};                 /*コマンド文字列のポインタ,引数文
字列のポインタ用*/
    97	
    98	  if(*line == '%')                             /*入力文字列の1文字目が%のとき*/
    99	    {
   100	      line++;
   101	      split(line, ret, ' ', 2);
   102	      exec_command(ret[0], ret[1]);
   103	    }
   104	  else if(profile_data_nitems < 10000)         /*入力がコマンドではなく,登録数が1
万件以下のとき*/
   105	    {
   106	      new_profile(&profile_data_store[profile_data_nitems++] ,line);
   107	    }
   108	  else
   109	    {
   110	      fprintf(stderr, "Warning: 10000件を超える名簿データは読み込めません.計%d件
の登録が中止されました.\n", i);
   111	      i++;
   112	    }
   113	}
   114	
   115	void exec_command(char *cmd, char *param)
   116	{
   117	
   118	  switch (*cmd) {
   119	  case 'Q': 
   120	    if(param == NULL)
   121	      {
   122		if(strcmp(cmd, "Q") == 0) cmd_quit("A");
   123		fprintf(stderr, "%%Q rと入力することで,確認メッセージなしで終了できます.
\n\n");
   124		cmd_quit(0);
   125	      }
   126	    else
   127	      {
   128		if(strcmp(param, "r") == 0) cmd_quit(param);
   129		else {
   130		  fprintf(stderr, "%%Q rと入力することで,確認メッセージなしで終了できます
.\n\n");
   131		  cmd_quit(0);
   132		}
   133	      }
   134	    break;
   135	
   136	  case 'C': cmd_check();  break;
   137	  case 'P':
   138	    if(strcmp(cmd, "P") == 0) cmd_print(param);
   139	    else fprintf(stderr, "コマンドの入力体裁が間違っています.処理を中止しました.
\nもしかして:\"%%%c %s\"\n\n", *cmd, (cmd + 1));
   140	    break;
   141	
   142	  case 'R':
   143	    if(strcmp(cmd, "R") == 0) cmd_read(param);
   144	    else fprintf(stderr, "コマンドの入力体裁が間違っています.処理を中止しました.
\nもしかして:\"%%%c %s\"\n\n", *cmd, (cmd + 1));
   145	    break;
   146	
   147	  case 'W':
   148	    if(strcmp(cmd, "W") == 0 ||strcmp(cmd, "WC") == 0) cmd_write(param, ',');
   149	    else if(strcmp(cmd, "WS") == 0) cmd_write(param, ';');
   150	    else if(strcmp(cmd, "WT") == 0) cmd_write(param, TAB);
   151	    else fprintf(stderr, "コマンドの入力体裁が間違っています.処理を中止しました.
\nもしかして:\"%%%c %c%c%s\"または\"%%%c%c %c%s\"\n\n", *cmd, *(cmd + 1), *(cmd + 2), (cm
d + 3), *cmd, *(cmd + 1), *(cmd + 2), (cmd + 3));
   152	    break;
   153	
   154	  case 'F':
   155	    if(strcmp(cmd, "F") == 0) cmd_find(param);
   156	    else fprintf(stderr, "コマンドの入力体裁が間違っています.処理を中止しました.
\nもしかして:\"%%%c %s\"\n\n", *cmd, (cmd + 1));
   157	    break;
   158	
   159	  case 'S':
   160	    if(strcmp(cmd, "S") == 0) cmd_sort(param);
   161	    else fprintf(stderr, "コマンドの入力体裁が間違っています.処理を中止しました.
\nもしかして:\"%%%c %s\"\n\n", *cmd, (cmd + 1));
   162	    break;
   163	
   164	  default: fprintf(stderr, "不明なコマンド\"%%%s\"です.処理を中止しました.\n\n",
 cmd); break;/*エラーメッセージを表示*/
   165	  }
   166	}
   167	
   168	void cmd_quit(char *param)
   169	{
   170	  char c = 65;
   171	
   172	  if(param != NULL)                            /*param部が存在している場合*/
   173	    if(*param == 'r')                          /*rオプションで確認メッセージなしで
終了*/
   174	      {
   175		printf("正常終了.\n\n");
   176		exit(0);
   177	      }
   178	 
   179	  while(1)
   180	    {
   181	      printf("終了しますか?(y/n)\n");            /*確認メッセージ*/
   182	      scanf("%c", &c);        /*いきなり改行文字を入力した場合,この下の処理は実行
しない*/
   183	      if(c != '\n')getchar(); /*scanfでの入力時に改行文字が残ってしまうため*/
   184	      if(c == 'y')
   185	  	{
   186	  	  printf("正常終了.\n\n");
   187		  exit(0);
   188		}
   189	      else if(c == 'n')
   190		{
   191		  printf("処理を中止しました.\n\n");
   192		  break;
   193		}
   194	    }
   195	}
   196	
   197	void cmd_check(void)                           /*名簿件数表示*/
   198	{
   199	  printf("%d profile(s)\n", profile_data_nitems);
   200	}
   201	
   202	void cmd_print(char *param)
   203	{
   204	  int a = 0;
   205	  int i = 0;                                   /*forループ用*/
   206	
   207	  /*atoi関数で正常に文字列をint値に変換できるかの確認を実施*/
   208	  if(param == NULL) a = 0;
   209	  if(param != NULL)
   210	    {
   211	      if(int_value_check(param))
   212		{
   213		  fprintf(stderr,"引数は数値である必要があります.処理を中止しました.\n\n
");
   214		  return;
   215		}
   216	      a = atoi(param);                         /*文字列をint型の値に変換*/
   217	    }
   218	
   219	  /*aの絶対値がprofile_data_nitemsより大きいときかa=0のとき*/
   220	  if(abs(a) >= profile_data_nitems|| a == 0) a = profile_data_nitems;
   221	
   222	  if(a > 0)                                    /*引数が正の整数のとき及び例外*/
   223	    {
   224	      for(i = 0; i < a; i++)
   225	        {
   226		  printf("Id    : %d\n", profile_data_store[i].id);
   227		  printf("Name  : %s\n", profile_data_store[i].name);
   228		  printf("Birth : %04d-%02d-%02d\n", profile_data_store[i].birthday.y, pro
file_data_store[i].birthday.m, profile_data_store[i].birthday.d);
   229		  printf("Addr. : %s\n",profile_data_store[i].address);
   230		  printf("Comm. : %s\n\n",profile_data_store[i].biko);
   231		}
   232	    }
   233	  else if(a < 0)                               /*引数が負の整数のとき*/
   234	    {
   235	      for(i = profile_data_nitems + a; i < profile_data_nitems; i++)
   236		{
   237		  printf("Id    : %d\n", profile_data_store[i].id);
   238		  printf("Name  : %s\n", profile_data_store[i].name);
   239		  printf("Birth : %04d-%02d-%02d\n", profile_data_store[i].birthday.y, pro
file_data_store[i].birthday.m, profile_data_store[i].birthday.d);
   240		  printf("Addr. : %s\n",profile_data_store[i].address);
   241		  printf("Comm. : %s\n\n",profile_data_store[i].biko);
   242		}
   243	    }
   244	}
   245	
   246	void cmd_read(char *param)
   247	{
   248	  char LINE[MAX_LINE] = {0};
   249	  FILE *fp;
   250	
   251	  if(param == NULL)
   252	    {
   253	      fprintf(stderr, "実行には引数が必要です.処理を中止しました.\n\n");
   254	      return;
   255	    }
   256	
   257	  if((fp = fopen(param, "r")) == NULL)         /*指定されたファイル名が存在しない
場合*/
   258	    {
   259	      fprintf(stderr, "\"%s\"を読み込めません.カレントディレクトリにファイルが存
在しないか,読み取り許可がない可能性があります.処理を中止しました.\n\n", param);
   260	      return;
   261	    }
   262	
   263	  while(get_line(fp, LINE))                    /*文字配列LINEに文字列を入力する*/
   264	    {
   265	      parse_line(LINE);                        /*入力文字列がある場合,構文解析を
行う*/
   266	    }
   267	
   268	  fclose(fp);
   269	}
   270	
   271	void cmd_write(char *param, char sep)
   272	{
   273	  int i;                                       /*forループ用*/
   274	  FILE *fp;
   275	
   276	  if(param == NULL)
   277	    {
   278	      fprintf(stderr, "実行には引数が必要です.処理を中止しました.\n\n");
   279	      return;
   280	    }
   281	
   282	  if((fp = fopen(param, "w")) == NULL)         /*指定されたファイル名が存在しない
場合*/
   283	    {
   284	      fprintf(stderr, "\"%s\"に書き込めません.書き込み許可がない可能性があります
.処理を中止しました.\n\n", param);
   285	      return;
   286	    }
   287	
   288	  /*CSV,SCSV,TSV形式で出力*/
   289	  for(i = 0; i < profile_data_nitems; i++)
   290	    {
   291	      fprintf(fp, "%d%c", profile_data_store[i].id, sep);
   292	      fprintf(fp, "%s%c", profile_data_store[i].name, sep);
   293	      fprintf(fp, "%d-%d-%d%c", profile_data_store[i].birthday.y, profile_data_sto
re[i].birthday.m, profile_data_store[i].birthday.d, sep);
   294	      fprintf(fp, "%s%c",profile_data_store[i].address, sep);
   295	      fprintf(fp, "%s\n",profile_data_store[i].biko);
   296	    }
   297	
   298	  fclose(fp);
   299	}
   300	
   301	void cmd_find(char *param)
   302	{
   303	  int i = 0;                                   /*forループ用*/
   304	  char num1[12];                               /*int値を文字列に変換する際に使用*/
   305	  char num2[36];                               /*int値を文字列に変換する際に使用*/
   306	  char num3[36];                               /*誕生日の0埋めなし用*/
   307	  struct profile *p;
   308	
   309	  if(param == NULL)
   310	    {
   311	      fprintf(stderr, "実行には引数が必要です.処理を中止しました.\n\n");
   312	      return;
   313	    }
   314	
   315	  for(i = 0; i < profile_data_nitems; i++)
   316	    {
   317	      p = &profile_data_store[i];
   318	
   319	      /*int値を文字列に変換して代入*/
   320	      sprintf(num1,"%d", p->id);
   321	      sprintf(num2,"%04d-%02d-%02d",(p->birthday).y, (p->birthday).m, (p->birthday
).d);
   322	      sprintf(num3,"%d-%d-%d",(p->birthday).y, (p->birthday).m, (p->birthday).d);
   323	
   324	      if(strcmp(param, num1) == 0 ||           /*ID比較*/
   325		 strcmp(param, p->name) == 0 ||        /*name比較*/
   326		 strcmp(param, num2) == 0 ||           /*birthday比較(0埋め)*/
   327		 strcmp(param, num3) == 0 ||           /*birthday比較(0無視)*/
   328		 strcmp(param, p->address) == 0 ||     /*address比較*/
   329		 strcmp(param, p->biko) == 0)          /*biko比較*/
   330		{                                      /*該当名簿情報表示*/
   331		  printf("Id    : %d\n", p->id);
   332		  printf("Name  : %s\n", p->name);
   333		  printf("Birth : %04d-%02d-%02d\n", (p->birthday).y, (p->birthday).m, (p-
>birthday).d);
   334		  printf("Addr. : %s\n",p->address);
   335		  printf("Comm. : %s\n\n",p->biko);
   336		 }
   337	    }
   338	}
   339	
   340	void cmd_sort(char *param)
   341	{
   342	  int a = 0;
   343	  int a_buff = 0;                              /*param変換後の値保持用*/  
   344	  int i1 = 0;                                  /*バブルソート用の右端を決める*/
   345	  int i2 = 0;                                  /*バブルソートの探索用*/
   346	  char num1[36];                               /*int値を文字列に変換する際に使用*/
   347	  char num2[36];                               /*int値を文字列に変換する際に使用*/
   348	  struct profile *sp1;
   349	  struct profile *sp2;
   350	  char *cp1;                                   /*任意文字列の先頭アドレスを保持*/
   351	  char *cp2;                                   /*任意文字列の先頭アドレスを保持*/
   352	
   353	  /*atoi関数で正常に文字列をint値に変換できるかの確認を実施*/
   354	  if(param == NULL)
   355	    {
   356	      fprintf(stderr, "実行には引数が必要です.処理を中止しました.\n\n");
   357	      return;
   358	    }
   359	  if(int_value_check(param))
   360	    {
   361	      fprintf(stderr, "引数は整数値である必要があります.処理を中止しました.\n\n"
);
   362	      return;
   363	    }
   364	  a_buff = atoi(param);
   365	
   366	  if(a_buff < 1 || a_buff > 5) /*引数のチェック*/
   367	    {
   368	      fprintf(stderr, "引数は1から5のいずれかの数値である必要があります.処理を中
止しました.\n\n");
   369	      return;
   370	    }
   371	
   372	  for(i1 = 0; i1 < profile_data_nitems; i1++)
   373	    {
   374	      for(i2 = 0; i2 < profile_data_nitems - i1 - 1; i2++)
   375		{
   376		  a = a_buff;                          /*例外対策処理用で毎度aを初期化*/
   377		  sp1 = &profile_data_store[i2];       /*名簿データの先頭アドレスを登録*/
   378		  sp2 = &profile_data_store[i2 + 1];   /*1つ後ろの名簿データの先頭アドレス
を登録*/
   379	
   380		  if(a == 2) /*氏名で並び換え*/
   381		    {
   382		      cp1 = sp1->name;
   383		      cp2 = sp2->name;
   384		      if(strcmp(cp1, cp2) == 0) a = 1; /*項目が一致していた場合は,IDの昇
順になるように並び変える*/
   385		    }
   386		  
   387		  if(a == 3) /*誕生日で並び換え*/
   388		    {
   389		      sprintf(num1, "%010d%010d%010d", (sp1->birthday).y, (sp1->birthday).
m, (sp1->birthday).d);
   390		      sprintf(num2, "%010d%010d%010d", (sp2->birthday).y, (sp2->birthday).
m, (sp2->birthday).d);
   391		      cp1 = num1;
   392		      cp2 = num2;
   393		      if(strcmp(cp1, cp2) == 0) a = 1; /*項目が一致していた場合は,IDの昇
順になるように並び変える*/
   394		    }
   395		  
   396		  if(a == 4) /*住所で並び換え*/
   397		    {
   398		      cp1 = sp1->address;
   399		      cp2 = sp2->address;
   400		      if(strcmp(cp1, cp2) == 0) a = 1; /*項目が一致していた場合は,IDの昇
順になるように並び変える*/
   401		    }
   402		  if(a == 5) /*備考で並び換え*/
   403		    {
   404		      cp1 = sp1->biko;
   405		      cp2 = sp2->biko;
   406		      if(strcmp(cp1, cp2) == 0) a = 1; /*項目が一致していた場合は,IDの昇
順になるように並び変える*/
   407		    }
   408	 	  
   409		  if(a == 1) /*IDで並び換え*/
   410		    {
   411		      sprintf(num1, "%010d", sp1->id);
   412		      sprintf(num2, "%010d", sp2->id);
   413		      cp1 = num1;
   414		      cp2 = num2;
   415		    }
   416	
   417		  if(strcmp(cp1, cp2) > 0) data_move(sp1, sp2); /*文字列比較*/
   418		}
   419	    }
   420	}
   421	
   422	void data_move(struct profile *sp1, struct profile *sp2)
   423	{
   424	  struct profile swap_data;
   425	  
   426	  /*sp1のデータをswap_dataに退避*/
   427	  swap_data.id = sp1->id;
   428	  strcpy(swap_data.name, sp1->name);
   429	  swap_data.birthday.y = (sp1->birthday).y;
   430	  swap_data.birthday.m = (sp1->birthday).m;
   431	  swap_data.birthday.d = (sp1->birthday).d;
   432	  strcpy(swap_data.address, sp1->address);
   433	  swap_data.biko = sp1->biko;
   434	
   435	  /*sp2のデータをsp1に移動*/
   436	  sp1->id = sp2->id;
   437	  strcpy(sp1->name, sp2->name);
   438	  (sp1->birthday).y = (sp2->birthday).y;
   439	  (sp1->birthday).m = (sp2->birthday).m;
   440	  (sp1->birthday).d = (sp2->birthday).d;
   441	  strcpy(sp1->address, sp2->address);
   442	  sp1->biko = sp2->biko;
   443	  
   444	  /*swap_dataをsp2に移動*/
   445	  sp2->id = swap_data.id;
   446	  strcpy(sp2->name, swap_data.name);
   447	  (sp2->birthday).y = (swap_data.birthday).y;
   448	  (sp2->birthday).m = (swap_data.birthday).m;
   449	  (sp2->birthday).d = (swap_data.birthday).d;
   450	  strcpy(sp2->address, swap_data.address);
   451	  sp2->biko = swap_data.biko;
   452	}
   453	
   454	void new_profile(struct profile *profile_p, char *line)
   455	{
   456	  char *ret[10];
   457	  char *ret2[4];                               /*誕生日の情報を分割し,その先頭ア
ドレスを保存*/
   458	  char sep;                                    /*入力文字列の要素を区切るのに使用*
/
   459	  char sep2 = '-';                             /*誕生日の入力文字列にあるハイフン
で区切るため,ハイフン*/
   460	  int max = 5;
   461	  int max2 = 4;
   462	  int birth_c = 0;
   463	  int MAX_BIKO = 0;                            /*備考の文字数カウント用*/
   464	  static int i = 0;                            /*入力項目の行番号監視*/
   465	  
   466	  i++;
   467	  sep = input_format_check(line);
   468	
   469	  if(sep == 0)                                 /*カンマ,セミコロン,タブのいずれ
でも区切れない場合*/
   470	    {
   471	      fprintf(stderr, "情報はID,名前,誕生日,住所,備考の順で入力される必要があ
り,カンマ区切り,セミコロン区切り,タブ区切りのいずれかの体裁である必要があります.処理を
中止しました(項目番号:%d).\n\n", i);
   472	      profile_data_nitems--;                   /*処理中止により,構造体に情報を書
き込まないため*/
   473	      return;
   474	    }
   475	  split(line, ret, sep, max);                  /*ID,名前などの情報を分割する*/
   476	
   477	  /*atoi関数で正常に文字列をint値に変換できるかの確認を実施*/
   478	  if(int_value_check(ret[0]))
   479	    {
   480	      fprintf(stderr,"IDの項目は数値である必要があります.処理を中止しました(項目
番号:%d).\n\n", i);
   481	      profile_data_nitems--;
   482	      return;
   483	    }
   484	
   485	  /*IDが負になる場合は,処理を中断する*/
   486	  if(atoi(ret[0]) < 0)
   487	    {
   488	      fprintf(stderr,"IDの項目は正の数値である必要があります.処理を中止しました(
項目番号:%d).\n\n", i);
   489	      profile_data_nitems--;
   490	      return;
   491	    }
   492	
   493	  birth_c = split(ret[2], ret2, sep2, max2);   /*誕生日の年,月,日を分割する*/
   494	  if(birth_c != 3)                             /*誕生日の年,月,日を正常に分割で
きない場合*/
   495	    {
   496	      fprintf(stderr, "誕生日は\"年-月-日\"の形で入力される必要があります.処理を
中止しました(項目番号:%d).\n\n", i); /*年,月,日に分割できない場合,処理を停止*/
   497	      profile_data_nitems--;                   /*処理中止により,構造体に情報を書
き込まないため*/
   498	      return;
   499	    }
   500	
   501	  /*atoi関数で正常に文字列をint値に変換できるかの確認を実施*/
   502	  if(int_value_check(ret2[0]) ||
   503	     int_value_check(ret2[1]) ||
   504	     int_value_check(ret2[2]) )
   505	    {
   506	      fprintf(stderr,"誕生年,月,日の項目は数値である必要があります.処理を中止し
ました(項目番号:%d).\n\n", i);
   507	      profile_data_nitems--;
   508	      return;
   509	    }
   510	
   511	  /*年月日の体裁チェック*/
   512	  if(day_format_check(atoi(ret2[0]), atoi(ret2[1]), atoi(ret2[2])))
   513	    {
   514	      fprintf(stderr,"その年月日は存在しません.処理を中止しました(項目番号:%d).\
n\n", i);
   515	      profile_data_nitems--;
   516	      return;
   517	    }
   518	
   519	  /*構造体への情報の書き込み処理*/
   520	  profile_p->id = atoi(ret[0]);                /*IDの書き込み*/
   521	  strncpy(profile_p->name, ret[1], 69);        /*名前の書き込み*/
   522	  (profile_p->birthday).y = atoi(ret2[0]);     /*誕生年の書き込み*/
   523	  (profile_p->birthday).m = atoi(ret2[1]);     /*誕生月の書き込み*/
   524	  (profile_p->birthday).d = atoi(ret2[2]);     /*誕生日の書き込み*/
   525	  strncpy(profile_p->address, ret[3], 69);     /*住所の書き込み*/
   526	
   527	  MAX_BIKO = strlen(ret[4]) + 1;               /*備考情報の文字数のカウント*/
   528	
   529	  profile_p->biko = (char *)malloc(sizeof(char)* MAX_BIKO); /*文字数分だけメモリ確
保*/
   530	  strncpy(profile_p->biko, ret[4], MAX_BIKO);  /*備考の書き込み*/
   531	}
   532	
   533	int int_value_check(char *str)
   534	{
   535	  if((*str >= 48 && *str <= 57) || *str == 43 || *str == 45) str++;/*1文字目は'-'
と'+'を許容*/
   536	  else return 1;
   537	
   538	  while(*str)                             /*入力文字列の終端に辿り着くまでループ*/
   539	    {
   540	      if(*str >= 48 && *str <= 57) str++; /*確認する文字が0~9の場合,次の文字を確
認*/
   541	      else return 1;                      /*確認する文字が0~9で無い場合,戻り値1*/
   542	    }
   543	  return 0;                               /*変換不可能文字列と判定した場合*/
   544	}
   545	
   546	char input_format_check(char *str)
   547	{
   548	  int com_c = 0;                          /*コンマの使用回数のカウント*/
   549	  int semi_c = 0;                         /*セミコロンの使用回数のカウント*/
   550	  int tab_c = 0;                          /*タブの使用回数のカウント*/
   551	
   552	  while(*str)                             /*入力文字列の終端に辿り着くまでループ*/
   553	    {
   554	      if(*str == ',') com_c++;            /*カンマ区切り*/
   555	      if(*str == ';') semi_c++;           /*セミコロン区切り*/
   556	      if(*str == TAB) tab_c++;            /*タブ区切り*/
   557	      str++;
   558	    }
   559	    
   560	  /*正しく区切れると見込まれる文字の文字コードを返す*/
   561	  if(com_c == 4) return 44;
   562	  if(semi_c == 4) return 59;
   563	  if(tab_c == 4) return 9;
   564	
   565	  return 0;                               /*どの文字でも区切ることができない場合(
例外)*/
   566	}
   567	
   568	int day_format_check(int y, int m, int d) /*存在する年月日の場合,戻り値0,それ以
外は戻り値1*/
   569	{  
   570	  switch(m){
   571	  case 1:
   572	    if(d >= 1 && d <= 31) return 0;
   573	    break;
   574	  case 2:
   575	    if((y % 4 == 0 && y % 100 != 0) || y % 400 == 0)/*閏年の場合,dを29日まで許可*
/
   576	      {
   577		if(d >= 1 && d <= 29) return 0;
   578	      }
   579	    else                                  /*そうでない場合は,dは28日まで*/
   580	      {
   581		if(d >= 1 && d <= 28) return 0;
   582	      }
   583	    break;
   584	  case 3:
   585	    if(d >= 1 && d <= 31) return 0;
   586	    break;
   587	  case 4:
   588	    if(d >= 1 && d <= 30) return 0;
   589	    break;
   590	  case 5:
   591	    if(d >= 1 && d <= 31) return 0;
   592	    break;
   593	  case 6:
   594	    if(d >= 1 && d <= 30) return 0;
   595	    break;
   596	  case 7:
   597	    if(d >= 1 && d <= 31) return 0;
   598	    break;
   599	  case 8:
   600	    if(d >= 1 && d <= 31) return 0;
   601	    break;
   602	  case 9:
   603	    if(d >= 1 && d <= 30) return 0;
   604	    break;
   605	  case 10:
   606	    if(d >= 1 && d <= 31) return 0;
   607	    break;
   608	  case 11:
   609	    if(d >= 1 && d <= 30) return 0;
   610	    break;
   611	  case 12:
   612	    if(d >= 1 && d <= 31) return 0;
   613	    break;
   614	  }
   615	  return 1;                               /*存在しない年月日である(例外)*/
   616	}
   617	
   618	int main(void)
   619	{
   620	  char LINE[MAX_LINE] = {0};              /*標準入力文字列(1行分)はmain関数で管理*
/
   621	
   622	  while(get_line(stdin, LINE))            /*文字配列LINEに文字列を入力する*/
   623	    {
   624	      parse_line(LINE);                   /*入力文字列がある場合,構文解析を行う*/
   625	    }
   626	  return 0;
   627	}
\end{Verbatim}

\subsection{自作の\texttt{string}型の変数で関数間で文字列を渡すプログラム}\label{funcstr}
プログラムのソースコード(26行)
\begin{Verbatim}[fontsize=\small, baselinestretch=0.8]
     1	#include <stdio.h>
     2	
     3	struct string
     4	{
     5		char a[50];
     6	};
     7	
     8	struct string func(void)
     9	{
    10		struct string fstr;
    11	
    12		printf("Input string :fstr\n");
    13		scanf("%s", &fstr.a[0]);
    14	
    15		return fstr;
    16	}
    17	
    18	int main(void)
    19	{
    20		struct string mstr;
    21	
    22		mstr = func();
    23		printf("Output string :mstr\n\"%s\"\n",&mstr.a[0]);
    24	
    25		return 0;
    26	}
\end{Verbatim}

%%%%%%%%%%%%%%%%%%%%%%%%%%%%%%%%%%%%%%%%%%%%%%%%%%%%%%%%%%%%%%%%%%%%%%
% 参考文献
%   実際に,参考にした書籍等の奥付などを見て書くこと.
%   本文で引用する際は,\cite{book:algodata}のように書けばよい.
%%%%%%%%%%%%%%%%%%%%%%%%%%%%%%%%%%%%%%%%%%%%%%%%%%%%%%%%%%%%%%%%%%%%%%
\begin{thebibliography}{99}
  \bibitem{book:algodata} 平田富雄,アルゴリズムとデータ構造,森北出版,1990.
  \bibitem{book:meikai} 林晴比古,明快入門C,SBクリエイティブ,2013
  \bibitem{www:label1} 入出力のリダイレクションとパイプ,http://web.sfc.keio.ac.jp/~manabu/command/contents/pipe.html,2020.05.14.
  \bibitem{www:label2} IT用語辞典E-Words ASCII文字コード,http://e-words.jp/p/r-ascii.html,2020.05.14
  \bibitem{www:label3} 10-3.ポインタと文字列,http://www9.plala.or.jp/sgwr-t/c/sec10-3.html,2020.05.14.
  \bibitem{www:label4} 4.3 ポインタ配列,http://cai3.cs.shinshu-u.ac.jp/sugsi/Lecture/c2/e\_04-03.html,2020.05.14.%アンダースコアはLaTeXの特殊文字にあたるため,\を挿入している.
  \bibitem{www:label5} strncpy,http://www9.plala.or.jp/sgwr-t/lib/strncpy.html,2020.5.21.
  \bibitem{www:label6} 文字列を数値に変換する - C の部屋,http://www.t-net.ne.jp/~cyfis/c/stdlib/atoi.html,2020.5.21.
  \bibitem{www:label7} IT用語辞典E-Words コアダンプ,http://e-words.jp/w/\%E3\%82\%B3\%E3\%82\%A2\%E3\%83\%80\%E3
\%83\%B3\%E3\%83\%97.html,2020.5.22.%'%'はLaTeXの特殊文字にあたるため,\を挿入している.
  \bibitem{www:label8} strlen,http://www9.plala.or.jp/sgwr-t/lib/strlen.html,2020.06.04
  \bibitem{www:label9} メモリ領域の動的確保,http://rainbow.pc.uec.ac.jp/edu/program/b1/programming-6.htm,2020.06.04
  \bibitem{www:label10} malloc,http://www9.plala.or.jp/sgwr-t/lib/malloc.html,2020.06.04
  %\bibitem{a} 著者名,書名,出版社,発行年.
  %\bibitem{b} WWWページタイトル,URL,アクセス日.
\end{thebibliography}

%--------------------------------------------------------------------%
%% 本文はここより上に書く(\begin{document}~\end{document}が本文である)
\end{document}
