% ファイル先頭から\begin{document}までの内容(プレアンブル)については,
% 基本的に { } の中を書き換えるだけでよい.
\documentclass[autodetect-engine,dvi=dvipdfmx,ja=standard,
               a4j,11pt]{bxjsarticle}

%%======== プレアンブル ============================================%%
% 用紙設定:指示があれば,適切な余白に設定しなおす
\RequirePackage{geometry}
\geometry{reset,a4paper}
\geometry{hmargin=25truemm,top=25truemm,bottom=25truemm,footskip=10truemm}
%\geometry{showframe} % 本文の"枠"を確認したければ,コメントアウト

% 設定:図の挿入
% http://www.edu.cs.okayama-u.ac.jp/info/tool_guide/tex.html#graphicx
\usepackage{graphicx}

% 設定:ソースコードの挿入
% http://www.edu.cs.okayama-u.ac.jp/info/tool_guide/tex.html#fancyvrb
\usepackage{fancyvrb}
\renewcommand{\theFancyVerbLine}{\texttt{\footnotesize{\arabic{FancyVerbLine}:}}}

%%======== レポートタイトル等 ======================================%%
% ToDo: 提出要領に従って,適切なタイトル・サブタイトルを設定する
\title{プログラミング演習1 \\
       第2回レポート}

% ToDo: 自分自身の氏名と学生番号に書き換える
\author{氏名: 重近 大智 (SHIGECHIKA, Daichi) \\
        学生番号: 09501527}

% ToDo: レポート課題等の指示に従って適切に書き換える
\date{出題日: 2020年04月30日 \\
      提出日: 2020年05月07日 \\
      締切日: 2020年05月13日 \\}  % 注:最後の\\は不要に見えるが必要.

%%======== 本文 ====================================================%%
\begin{document}
\maketitle
% 目次つきの表紙ページにする場合はコメントを外す
%{\footnotesize \tableofcontents \newpage}

%% 本文は以下に書く.課題に応じて適切な章立てを構成すること.
%% 章=\section,節=\subsection,項=\subsubsection である.

%%%%%%%%%%%%%%%%%%%%%%%%%%%%%%%%%%%%%%%%%%%%%%%%%%%%%%%%%%%%%%%%%%%%%%
\section{概要} \label{sec:abstract}
%%%%%%%%%%%%%%%%%%%%%%%%%%%%%%%%%%%%%%%%%%%%%%%%%%%%%%%%%%%%%%%%%%%%%%

本演習では,名簿管理機能を有するプログラムを,C言語で作成する.
このプログラムは,標準入力から「ID, 氏名, 誕生日, 住所, 備考」からなる
コンマ区切り形式(CSV形式)の名簿データを受け付けて,それらをメモリ中に登録する機能を持つ.
ただし,\%で始まる入力行はコマンド入力と解釈し,登録してあるデータを表示したり整列したりする機能も持つ.

本レポートでは,演習中に取り組んだ課題として,
以下の課題1から課題3についての内容を報告する.
%
\begin{description}
  \item[課題1] 文字列操作の基礎:\verb|subst|関数と\verb|split|関数の実装
  \item[課題2] 構造体や配列を用いた名簿データの定義
  \item[課題3] 標準入力の取得と構文解析
  %追記する場合は,本文の記述と矛盾しないように注意すること.
\end{description}
%
また,取り組んだ課題のうち,特に以下の課題については,詳細な考察を行った.
%
\begin{description}
  \item[課題1] 文字列操作の基礎:\verb|subst|関数と\verb|split|関数の実装
  \item[課題3] 標準入力の取得と構文解析
  %追記する場合は,本文の記述と矛盾しないように注意すること.
\end{description}


%%%%%%%%%%%%%%%%%%%%%%%%%%%%%%%%%%%%%%%%%%%%%%%%%%%%%%%%%%%%%%%%%%%%%%
\section{プログラムの作成方針}\label{sec:plan}
%%%%%%%%%%%%%%%%%%%%%%%%%%%%%%%%%%%%%%%%%%%%%%%%%%%%%%%%%%%%%%%%%%%%%%

本演習で作成したプログラムが満たすべき要件と仕様として,
「(1) 基本要件」と「(2) 基本仕様」を示す.

\subsubsection*{(1) 基本要件}

\begin{enumerate}
  \setlength{\parskip}{0em} \setlength{\itemsep}{0.25em}  % <-- この一行は項目間の調整
    \item プログラムは,その実行中,少なくとも10,000件の名簿データをメモリ中に保持できるようにすること.
    \item 名簿データは,・・・・
    \item プログラムとしての動作や名簿データの管理のために,・・・
    \begin{enumerate} \setlength{\parskip}{0em} \setlength{\itemsep}{0.25em}
        \item プログラムの正常な終了
        \item 登録された・・・
        % サンプルのため省略
    \end{enumerate}
    \item 標準入力からのユーザ入力を通して,,,      % サンプルのため省略
\end{enumerate}

\subsubsection*{(2) 基本仕様}

\begin{enumerate}
  \setlength{\parskip}{0em} \setlength{\itemsep}{0.25em}  % <-- この一行は項目間の調整
    \item 名簿データは,コンマ区切りの文字列(\textbf{CSV入力}と呼ぶ)で表されるものとし,%
          図\ref{fig:csvdata}に示したようなテキストデータを処理できるようにする.%
          %CSV入力の詳細は,\ref{sec:sepc_csv}節に示す.  % <-- この1文は,レポートでは不要
    \item コマンドは,\%で始まる文字列(\textbf{コマンド入力}と呼ぶ)とし,表\ref{tab:commands}にあげたコマンドをすべて実装する
    \item 1つの名簿データは,C言語の構造体 (\texttt{struct}) を用いて,・・・   % サンプルのため省略
    % サンプルのため省略.仕様書の footnote は書かなくてよい.
\end{enumerate}

%========================= EXAMPLE CSV ================================%
\begin{figure}[b]
\centering % この1行はbegin~endの中を中央寄せにする,というコマンド
% Verbatim environment
% プリアンブルで \usepackage{fancyvrb} が必要.
%   - frame         枠で囲う.single なら,四角で囲われる
%   - xleftmargin   枠の左の余白
%   - xrightmargin  枠の右の余白
%   - gobble        先頭n文字を無視.インデントを無視したい場合に利用する.
%   - fontsize      フォントサイズ指定
%   - numbers       行番号を表示.left なら左に表示.
%   - firstnumber   最初の行番号を指定
\begin{Verbatim}[frame=single, xleftmargin=5mm, xrightmargin=5mm, gobble=4,
                 fontsize=\small, numbers=left, firstnumber=1]
    5100046,The Bridge,1845-11-2,14 Seafield Road Longman Inverness,SEN Unit 2.0 Open
    5100127,Bower Primary School,1908-1-19,Bowermadden Bower Caithness,01955 641225 ...
    5100224,Canisbay Primary School,1928-7-5,Canisbay Wick,01955 611337 Primary 56 3...
    5100321,Castletown Primary School,1913-11-4,Castletown Thurso,01847 821256 01847...
\end{Verbatim}
    \caption{名簿データのCSV入力形式の例.1行におさまらないデータは...で省略した.}
    \label{fig:csvdata}
\end{figure}
%========================= EXAMPLE CSV ================================%

%========================= COMMAND LIST ================================%
\begin{table}[b]  %表の位置は原則として t または b.h や H は使わない.
\centering % この1行はbegin~endの中を中央寄せにする,というコマンド
    \caption{実装するコマンド}
    \label{tab:commands}
    \begin{tabular}{|l|ll|l|}
    \hline
    \textbf{コマンド} & \textbf{意味} &        & \textbf{備考} \\
    \hline\hline
    \verb|%Q|   & 終了                & (Quit)  & \\
    \hline
    \verb|%P n| & 先頭から$n$件表示   & (Print) & $n$が$0$ $\to$ 全件表示,        \\
                &                     &         & $n$が負 $\to$ 後ろから$-n$件表示 \\
    \hline
    *****       &(サンプルのため     & 省略)    & \\
    \hline
    \end{tabular}
\end{table}
%========================= COMMAND LIST ================================%


%%%%%%%%%%%%%%%%%%%%%%%%%%%%%%%%%%%%%%%%%%%%%%%%%%%%%%%%%%%%%%%%%%%%%%
\section{プログラムの説明}\label{sec:capp}
%%%%%%%%%%%%%%%%%%%%%%%%%%%%%%%%%%%%%%%%%%%%%%%%%%%%%%%%%%%%%%%%%%%%%%

プログラムリストは\ref{sec:makep}章に添付している.プログラムは全部で70行からなる.
以下では,\ref{sec:abstract}節の課題ごとに,プログラムの主な構造について説明する.

%--------------------------------------------------------------------%
\subsection{文字列操作の基礎:\texttt{subst}関数と\texttt{split}関数の実装}
%--------------------------------------------------------------------%
% Tips:
% - subsectionなどでは verbが使えない.texttt を使うとよい.
% - 最初の段落で,この節で説明する関数等の名前や場所などを説明する.
% - 以降の段落で,最初の段落で示した,各関数等に関する詳細を説明する.

まず,汎用的な文字列操作関数として,
\verb|subst()|関数を6--20行目で宣言し,
\verb|split()|関数を22--39行目で宣言している.
また,これらの関数で利用するために,\verb|<stdio.h>|というヘッダファイルをインクルードする.

\verb|subst(str, C1, C2)|関数は,
\verb|str|が指す文字列中の,文字\verb|C1|を文字\verb|C2|に置き換える.
プログラム中では,\verb|get_line()|関数内の\verb|fgets()|関数で文字列の入力を受けるとき,末尾に付く改行文字を\verb|NULL|文字で置き換えるために使用している.呼び出し元には,文字を置き換えた回数を戻り値として返す.

\verb|split(str, ret[], sep, max)|関数は,他関数から渡された文字列中に文字変数\verb|sep|の文字に一致する文字があった場合,該当文字を\verb|NULL|文字で置き換え,該当文字の次の文字が格納されているメモリのアドレスを\verb|ret[]|に書き込む.なお,\verb|ret[0]|に格納されるアドレスの値は,\verb|split()|関数が呼び出された際の\verb|str|の値である.以降,\verb|sep|の文字に一致する文字があった場合,\verb|ret[]|の添字を1ずつ増やしながらアドレスを格納していく.呼び出し元には,アドレスが格納されている\verb|ret[]|の内,添字が最も大きいものの添字を戻り値として返す.

%--------------------------------------------------------------------%
\subsection{構造体や配列を用いた名簿データの定義}
%--------------------------------------------------------------------%

本名簿管理プログラムでは,構造体の配列を名簿データとして扱う.
18--27行目で,\verb|date|構造体を定義し,
29--48行目で,\verb|profile|構造体を定義している.
この・・・・が,名簿データ1つに相当する.
そして,xxx行目の\verb|xxxxx|変数で,全名簿データを管理し,
xxx行目の\verb|xxxxx|変数で,名簿データの個数を管理する.

\verb|date|構造体の定義にあたっては,・・・(以降,サンプルのため,省略)

・・・(サンプルのため,省略)

%--------------------------------------------------------------------%
\subsection{標準入力の取得と構文解析}
%--------------------------------------------------------------------%

標準入力を取得するための\verb|get_line()|関数は41--47行目で宣言している.標準入出力のため,\verb|<stdio.h>|というヘッダファイルをインクルードする.

\verb|get_line(line)|関数は,標準入力\verb|stdin|を\verb|fgets()|関数で取得し,1024文字以上を越えた場合は,次の行として処理を行うことでバッファオーバーランを防止している.標準入力が\verb|NULL|または,\verb|ESC|キーを1文字目に入力した場合は,呼び出し元に戻り値$0$を返す.それ以外の場合,\verb|subst()|関数で末尾の改行文字を\verb|NULL|文字に置き換えた後,呼び出し元に戻り値$1$を返す.


%%%%%%%%%%%%%%%%%%%%%%%%%%%%%%%%%%%%%%%%%%%%%%%%%%%%%%%%%%%%%%%%%%%%%%
\section{プログラムの使用法と実行結果}\label{sec:howresult}
%%%%%%%%%%%%%%%%%%%%%%%%%%%%%%%%%%%%%%%%%%%%%%%%%%%%%%%%%%%%%%%%%%%%%%

本プログラムは名簿データを管理するためのプログラムである.
CSV形式のデータと \% で始まるコマンドを標準入力から受け付け,
処理結果を標準出力に出力する.
入力形式の詳細については,\ref{sec:plan}節で説明した.

% 実行結果を得るために利用した,実行環境のOSを必ず書くこと.
プログラムは,Cent OS 7.6.1810(Core) で動作を確認しているが,
一般的な UNIX で動作することを意図している.
なお,以降の実行例における,行頭の\verb|$|記号は,
Cent OS 7.6.1810(Core) におけるターミナルのプロンプトである.

まず,\verb|gcc|でコンパイルすることで,プログラムの実行ファイルを生成する.
ここで,\verb|-Wall|とは通常は疑わしいものとみなされることのない構文に関して警告を出力するためのオプションであり,
\verb|-o|とは出力ファイル名を指定するオプションである.
これらのオプションをつけることで,疑わしい構文を発見し,任意の出力ファイル名を指定することができる.

{\fontsize{10pt}{11pt} \selectfont
 \begin{verbatim}
   $ gcc -Wall -o program1 program1.c
 \end{verbatim}
}
%% 注:行送りの変更は"指定箇所を含む段落”に効果があらわれる.
%%     fontsizeコマンドを用いて,行送りを変える場合は,
%%     その {...} の前後に空白行を入れ,段落を変えるようにすること.
%%     なお,行先頭がコメントから始まる行は空白行とは扱われない.

次に,プログラムを実行する.
以下の実行例は,プログラム実行中の動作例を模擬するため,
任意の\verb|csv|ファイルを標準入力のリダイレクションにより与えることで,実行する例を示している.
通常の利用においては,キーボードから文字列を入力してもよい.

{\fontsize{10pt}{11pt} \selectfont
 \begin{verbatim}
   $ ./program1 < csvdata.csv
 \end{verbatim}
}

プログラムの出力結果として,CSVデータの各項目が読みやすい形式で出力される.
例えば,下記の \verb|csvdata.csv| に対して,

{\fontsize{10pt}{11pt} \selectfont
 \begin{verbatim}
  0,Takahashi Kazuyuki,1977-04-27,3,Saitama,184,78
  10,Honma Mitsuru,1972-08-25,2,Hokkaidou,180,78
  %S3
  %P0
 \end{verbatim}
}

\noindent % noindentとはここでは段落を変えない(一字下げをしない)というコマンド.
以下のような出力が得られる.

{\fontsize{10pt}{11pt} \selectfont
 \begin{verbatim}
  code:   10
  name:   Honma Mitsuru
  bday:   1972/08/25
  type:   2
  home:   Hokkaidou
  height: 180
  weight: 78

  code:   0
  name:   Takahashi Kazuyuki
  bday:   1977/04/27
  type:   3
  home:   Saitama
  height: 184
  weight: 78
 \end{verbatim}
}

まず,入力データについて説明する.
入力中の最初の2行で,2つのCSVデータを登録している.
CSVデータは,・・・で示したように,7つの項目からなる.
% 注:上1行の「表1」の部分は,\refを用いて適切に参照すること.
3行目の\%S3 は,これまでの入力データを3番目の項目(生年月日)で
ソートすることを示している.
4行目の\%P0 は,入力した項目の全ての項目 ($1$--$7$) を
表示することを示している.

(※サンプルのため省略)



%%%%%%%%%%%%%%%%%%%%%%%%%%%%%%%%%%%%%%%%%%%%%%%%%%%%%%%%%%%%%%%%%%%%%%
\section{考察}
%%%%%%%%%%%%%%%%%%%%%%%%%%%%%%%%%%%%%%%%%%%%%%%%%%%%%%%%%%%%%%%%%%%%%%

\ref{sec:capp}章のプログラムの説明,および,\ref{sec:howresult}章の使用法と実行結果から,
演習課題として作成したプログラムが,
\ref{sec:abstract}章で述べた基本要件と基本仕様のいずれも満たしていることを示した.
ここでは,個別の課題のうち,以下の3つの項目について,考察を述べる.
% 注:\ref コマンドを用いて適切に参照すること.

\begin{enumerate}
\setlength{\parskip}{2pt} \setlength{\itemsep}{2pt}
    \item 文字列操作の基礎:\texttt{subst}関数と\texttt{split}関数の実装
    \item 標準入力の取得と構文解析
    \item ・・・(\textbf{サンプルのため,省略})
\end{enumerate}

%--------------------------------------------------------------------%
\subsection{「文字列操作の基礎:\texttt{subst}関数と\texttt{split}関数の実装」に関する考察}
%--------------------------------------------------------------------%

\verb|subst()|関数が,他関数から文字列の配列を受け取ることを想定して,ポインタを使用して他関数内の配列を参照できるようにした.また,入力文字列の途中に\verb|NULL|文字が出現することは標準入力では起こりにくいため,文字置き換えのループ処理の継続条件に\verb|NULL|文字を用いることで,確実に入力文字列の終端である\verb|NULL|文字手前まで文字の置き換えを行えるようにした.\verb|c1|と\verb|c2|に同じ文字が入力された場合,文字の置き換え処理は実質行われないに等しい.12行目にbreak文を書くことにより,文字の置き換えと置き換えられた文字のカウントを行わず,処理の簡略化及び高速化ができた.


%--------------------------------------------------------------------%
\subsection{「標準入力の取得と構文解析」に関する考察}
%--------------------------------------------------------------------%

\verb|fgets()|関数で標準入力を取得する際,直接入力では標準入力を\verb|NULL|にすることができず,改行文字が入ってしまうため,\verb|ESC|キーを1文字目に入力することにより,入力待ちを終了できるように,44行目に戻り値の条件を追加した.\verb|fgets()|関数により取得した文字列は,終端が改行文字になってしまうため,\verb|subst()|関数を呼び出し,改行文字を\verb|NULL|文字に置き換える処理も含めている.この処理を行うことで,\verb|subst()|関数のループ処理の継続条件,\verb|split|関数のループ処理の継続条件や\verb|printf()|関数による文字列の出力などで文字列が扱いやすくなった.

\subsection{「・・・(\textbf{サンプルのため,省略})」に関する考察}

(※サンプルのため省略)



%%%%%%%%%%%%%%%%%%%%%%%%%%%%%%%%%%%%%%%%%%%%%%%%%%%%%%%%%%%%%%%%%%%%%%
\section{感想}\label{sec:review}
%%%%%%%%%%%%%%%%%%%%%%%%%%%%%%%%%%%%%%%%%%%%%%%%%%%%%%%%%%%%%%%%%%%%%%

(※サンプルのため省略)
%プログラム全体を通して表記のゆれが少なくなるよう,ポインタの指し先をずらすために変数\verb|i|を使用するようにした.

%%%%%%%%%%%%%%%%%%%%%%%%%%%%%%%%%%%%%%%%%%%%%%%%%%%%%%%%%%%%%%%%%%%%%%
\section{作成したプログラム}\label{sec:makep}
%%%%%%%%%%%%%%%%%%%%%%%%%%%%%%%%%%%%%%%%%%%%%%%%%%%%%%%%%%%%%%%%%%%%%%

作成したプログラムを以下に添付する.
なお,\ref{sec:abstract}章に示した課題については,
\ref{sec:howresult}章で示したようにすべて正常に動作したことを付記しておく.

% Verbatim environment
% プリアンブルで \usepackage{fancyvrb} が必要.
%   - numbers           行番号を表示.left なら左に表示.
%   - xleftmargin       枠の左の余白.行番号表示用に余白を与えたい.
%   - numbersep         行番号と枠の間隙 (gap).デフォルトは 12 pt.
%   - fontsize          フォントサイズ指定
%   - baselinestretch   行間の大きさを比率で指定.デフォルトは 1.0.
\verb|subst|,\verb|split|,\verb|get_line|関数を含むプログラムのソースコード(70行)\label{func:subst}
\begin{Verbatim}[
                    fontsize=\small, baselinestretch=0.8]
     1	#include <stdio.h>
     2	
     3	#define ESC 27                                             /*文字列ESCをESCのASCII
コードで置換*/
     4	#define MAX_LINE 1025                                      /*文字配列LINEの
最大入力数の指定用*/
     5	
     6	int subst(char *str, char c1, char c2)
     7	{
     8	  int i;                                                   /*forループ用*/
     9	  int c = 0;                                               /*置き換えた文字数の
カウント用*/
    10	  for(i = 0; *(str + i) != '\0'; i++)                      /*入力文字列の終端に辿り
着くまでループ*/
    11	    {
    12	      if(c1 == c2) break;                                  /*見た目上文字列に変化が
ないとき*/
    13	      if(*(str + i) == c1)                                 /*(str + i)の文字がc1の
文字と同じとき*/
    14		{
    15		  *(str + i) = c2;                                 /*(str + i)の文字をc2の
文字に置き換える*/
    16		  c++;                                             /*置き換えた文字を数える*/
    17		}
    18	    }
    19	  return c;                                                /*置き換えた文字数を戻り値
とする.*/
    20	}
    21	
    22	int split(char *str, char *ret[], char sep, int max)
    23	{
    24	  int i;                                                   /*forループ用*/
    25	  int c = 0;                                               /*ポインタの配列の指定用*/
    26	
    27	  ret[0] = str;                                            /*ret[0]にstrの先頭アドレス
を代入*/
    28	
    29	  for(i = 0; *(str + i) != '\0'&& c < max; i++)            /*cがmaxより小さいか,入力
文字列の終端に辿り着いていないときループ*/
    30	    {
    31	      if(*(str + i) == sep)                                /*(str + i)がsepのとき*/
    32		{
    33		  *(str + i) = '\0';                               /*(str + i)にNULLを代入*/
    34		  c++;
    35		  ret[c] = str + (i + 1);                          /*ret[c]にNULL文字の"次の"
アドレスを代入*/
    36		}
    37	    }
    38	  return c;                                                /*文字列をいくつに分割したか
を戻り値とする*/
    39	}
    40	
    41	int get_line(char *line)
    42	{
    43	  if(fgets(line, MAX_LINE, stdin) == NULL) return 0;       /*入力文字列が空のとき,
0を戻り値とする.入力文字列は1024文字*/
    44	  if(*line == ESC) return 0;                               /*直接入力のとき,入力文字列
を空にできないため,ESCキーの単独入力により0を戻り値とする*/
    45	  subst(line, '\n', '\0');                                 /*subst関数により,入力の
改行文字を終端文字に置き換える*/
    46	  return 1;                                                /*入力文字列が存在したとき,
1を戻り値とする*/
    47	}
    48	
    49	int main(void)
    50	{
    51	  char line[MAX_LINE] = {0};                               /*入力文字列は最大1024文字*/
    52	  char *ret[10];
    53	  char sep = ',';                                          /*csvファイルからの入力を
想定しているため,カンマ*/
    54	  int max = 10;
    55	  int c, i, a = 1;
    56	  
    57	  while(get_line(line))                                    /*get_line関数を呼び出し,
戻り値が0ならループを終了*/
    58	    {
    59	      printf("line number:%d\n", a);
    60	      printf("input:\"%s\"\n", line);
    61	      c = split(line, ret, sep, max);                      /*split関数を呼び出す*/
    62	      for(i = 0;i <= c; i++)
    63		{
    64		  printf("split[%d]:\"%s\"\n", i, ret[i]);
    65		}
    66	      printf("\n\n");                                      /*見やすさのために改行*/
    67	      a++;
    68	    }
    69	  return 0;
    70	}

\end{Verbatim}


%%%%%%%%%%%%%%%%%%%%%%%%%%%%%%%%%%%%%%%%%%%%%%%%%%%%%%%%%%%%%%%%%%%%%%
% 参考文献
%   実際に,参考にした書籍等の奥付などを見て書くこと.
%   本文で引用する際は,\cite{book:algodata}のように書けばよい.
%%%%%%%%%%%%%%%%%%%%%%%%%%%%%%%%%%%%%%%%%%%%%%%%%%%%%%%%%%%%%%%%%%%%%%
\begin{thebibliography}{99}
  \bibitem{book:algodata} 平田富雄,アルゴリズムとデータ構造,森北出版,1990.
  \bibitem{book:label2} 著者名,書名,出版社,発行年.
  \bibitem{www:label3} WWWページタイトル,URL,アクセス日.
\end{thebibliography}

%--------------------------------------------------------------------%
%% 本文はここより上に書く(\begin{document}~\end{document}が本文である)
\end{document}
